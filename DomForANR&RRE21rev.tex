%% This document created by Scientific Word (R) Version 3.0


\documentclass[12pt]{article}
%%%%%%%%%%%%%%%%%%%%%%%%%%%%%%%%%%%%%%%%%%%%%%%%%%%%%%%%%%%%%%%%%%%%%%%%%%%%%%%%%%%%%%%%%%%%%%%%%%%%%%%%%%%%%%%%%%%%%%%%%%%%%%%%%%%%%%%%%%%%%%%%%%%%%%%%%%%%%%%%%%%%%%%%%%%%%%%%%%%%%%%%%%%%%%%%%%%%%%%%%%%%%%%%%%%%%%%%%%%%%%%%%%%%%%%%%%%%%%%%%%%%%%%%%%%%
\usepackage{amsfonts}
\usepackage{amsmath}
\usepackage{amssymb}
\usepackage{graphicx}
\usepackage{amsopn}
\usepackage{amsthm}

\setcounter{MaxMatrixCols}{10}
%TCIDATA{OutputFilter=LATEX.DLL}
%TCIDATA{Version=5.00.0.2606}
%TCIDATA{<META NAME="SaveForMode" CONTENT="1">}
%TCIDATA{BibliographyScheme=Manual}
%TCIDATA{Created=Wednesday, June 11, 2008 12:02:54}
%TCIDATA{LastRevised=Sunday, March 13, 2011 13:41:49}
%TCIDATA{<META NAME="GraphicsSave" CONTENT="32">}
%TCIDATA{<META NAME="DocumentShell" CONTENT="Standard LaTeX\Standard LaTeX Article(mine)">}
%TCIDATA{Language=American English}
%TCIDATA{CSTFile=LaTexArt(mine).cst}

\setlength{\evensidemargin}{0in}
\setlength{\oddsidemargin}{0in}
\setlength{\textwidth}{6.25in}
\setlength{\textheight}{8.5in}
\setlength{\topmargin}{0in}
\setlength{\headheight}{0in}
\setlength{\headsep}{0in}
\setlength{\itemsep}{0pt}
\renewcommand{\topfraction}{.9}
\renewcommand{\textfraction}{.1}
\setlength{\parskip}{\smallskipamount}
\newtheorem{theorem}{Theorem}[section]
\newtheorem{axiom}[theorem]{Axiom}
\newtheorem{claim}[theorem]{Claim}
\newtheorem{conclusion}[theorem]{Conclusion}
\newtheorem{condition}[theorem]{Condition}
\newtheorem{conjecture}[theorem]{Conjecture}
\newtheorem{corollary}[theorem]{Corollary}
\newtheorem{lemma}[theorem]{Lemma}
\newtheorem{notation}[theorem]{Notation}
\newtheorem{proposition}[theorem]{Proposition}
\theoremstyle{definition}
\newtheorem{definition}[theorem]{Definition}
\newtheorem{example}[theorem]{Example}
\newtheorem{exercise}[theorem]{Exercise}
\newtheorem{problem}[theorem]{Problem}
\newtheorem{question}[theorem]{Question}
\newtheorem{remark}[theorem]{Remark}
\input{tcilatex}

\begin{document}

\title{Domination, forcing, array nonrecursiveness and relative recursive
enumerability\addtocounter{footnote}{1}}
\author{Mingzhong Cai\thanks{%
Partially supported by NSF Grants DMS-0554855 and DMS-0852811.} \\
%EndAName
Department of Mathematics\\
Cornell University\\
Ithaca NY 14853\addtocounter{footnote}{-2} \and Richard A. Shore\thanks{%
Partially supported by NSF Grants DMS-0554855 and DMS-0852811 and John
Templeton Foundation Grant 13408.} \\
%EndAName
Department of Mathematics\\
Cornell University\\
Ithaca NY 14853}
\maketitle

\begin{abstract}
We present some abstract theorems showing how domination properties
equivalent to being $\overline{\mathbf{GL}}_{2}$ or array nonrecursive can
be used to construct sets generic for different notions of forcing. These
theorems are then applied to give simple proofs of some known results. We
also give a direct uniform proof of a recent result of Ambos-Spies, Ding,
Wang and Yu [2009] that every degree above any in $\overline{\mathbf{GL}}%
_{2} $ is recursively enumerable in a 1-generic degree strictly below it.
Our major new result is that every array nonrecursive degree is r.e.\ in
some degree strictly below it. Our analysis of array nonrecursiveness and
construction of generic sequences below $\mathbf{ANR}$ degrees also reveal a
new level of uniformity in these types of results.
\end{abstract}

\section{Introduction}

The motivations for the work presented here were twofold. The first was the
similarity between certain constructions of degrees below a nonzero
recursively enumerable one and the analogous ones for degrees that are $%
\overline{\mathbf{GL}_{2}}$ or $\mathbf{ANR}$ (array nonrecursive). (A
Turing degree $\mathbf{a}$ is in $\mathbf{GL}_{2}$ if $\mathbf{a}^{\prime
\prime }=(\mathbf{a\vee 0}^{\prime })^{\prime }$. If $\mathbf{a}\notin 
\mathbf{GL}_{2}$ we say $\mathbf{a}\in $ $\overline{\mathbf{GL}_{2}}$. An
equivalent condition is that for every function $g\leq _{T}\mathbf{a\vee 0}%
^{\prime }$, there is an $f\leq _{T}\mathbf{a}$ which is not dominated by $g$%
, i.e. $\exists ^{\infty }x(g(x)<f(x))$ (see Lerman [1983, IV.3.4]).) A
degree $\mathbf{a}$ is \emph{array nonrecursive} ($\mathbf{ANR}$) if, for
every function $g\leq _{wtt}0^{\prime }$, there is an $f\leq _{T}\mathbf{a}$
which is not dominated by $g$ (Downey, Jockusch and Stob [1996], hereafter
[DJS]). In particular, there are theorems from Shore [1981, Lemma 4.2] and
Shore [2007, Theorem 4.1] about coding sets which are $\Sigma _{3}^{A}$
below $\mathbf{a}$ for $A\in \mathbf{a}$ when $\mathbf{a}$ is either r.e.\
and nonrecursive or $\mathbf{ANR}$. These results played crucial roles in
the proofs, respectively, that the theory of $\mathcal{D}(\leq \mathbf{0}%
^{\prime })$, the degrees below $\mathbf{0}^{\prime }$, is recursively
isomorphic to that of true (first order) arithmetic and that the Turing jump
operator is directly definable in any jump ideal containing $\mathbf{0}%
^{(\omega )}$, the degree of the truth set of (first order) arithmetic. Both
theorems were proved by fairly complicated but in some ways similar
constructions. The first used what is called r.e.\ permitting and the second 
$\mathbf{ANR}$ permitting. Both, like many constructions below r.e.,\ $%
\mathbf{ANR}$ or $\overline{\mathbf{GL}_{2}}$ degrees, depend on domination
properties of the given degree to carry out a type of forcing argument
(meeting various dense sets) in a type of priority construction.

Thus it seemed that these and other results could be simplified by proving
that $\mathbf{ANR}$ (and so \emph{a fortiori }$\overline{\mathbf{GL}_{2}}$)
degrees are \emph{relatively recursively enumerable}, $\mathbf{RRE}$, i.e.
recursively enumerable in some degree strictly below them. The point here is
that, if this were true, then the need for some of the separate proofs for $%
\mathbf{ANR}$ or $\overline{\mathbf{GL}_{2}}$ degrees could be eliminated by
simply citing the corresponding ones for r.e. degrees. Moreover, it seemed
desirable to formulate a general theorem about meeting classes of dense sets
for specified notions of forcing based on the relevant domination properties
characterizing the $\mathbf{ANR}$ and $\overline{\mathbf{GL}_{2}}$ degrees
that would unify the various constructions exploiting these properties.

The second motivation for this work was the paper of Ambos-Spies, Ding, Wang
and Yu [2009], hereafter [ASDWY]. They proved the following:

\begin{theorem}
\label{asdwy}\emph{([ASDWY, Theorem 1.5]) }Every $\mathbf{a\in \overline{%
\mathbf{GL}_{2}}}$ is $\mathbf{RRE}$ and, in fact,every $\mathbf{b}$ above
any $\mathbf{a\in \overline{\mathbf{GL}_{2}}}$ is r.e.\ in some 1-generic
degree $\mathbf{c<b}$.
\end{theorem}

[ASDWY] then raised a number of interesting questions asking for
characterizations of the degrees $\mathbf{a}$ such that every $\mathbf{b\geq
a}$ is $\mathbf{RRE}$ and the relation between being $\mathbf{RRE}$ and (the
apparently stronger) property of being r.e.\ in a strictly smaller 1-generic
degree. (A set $G$ is 1-generic if, for every r.e.\ set of binary strings $S$%
, there is a binary string $\sigma \subset G$ such that either $\sigma \in S$
or $(\forall \tau \supseteq \sigma )(\tau \notin S)$. A degree is 1-generic
if it contains a 1-generic set.)

We present in \S \ref{domsec} an analysis of the domination properties
characterizing array nonrecursiveness that provide an appropriate definition
for a relativized version of the notion. The analysis also proves that $%
\mathbf{ANR}$ degrees satisfy a stronger domination property with greater
uniformity than previously established. This property is closer to that
characterizing $\overline{\mathbf{GL}_{2}}$ and allows us to give a
framework for a general theorem about meeting dense sets recursively in
either $\overline{\mathbf{GL}_{2}}$ or $\mathbf{ANR}$ degrees. Even in the $%
\overline{\mathbf{GL}_{2}}$ case, our Theorem \ref{dom} is more general than
the ones in the literature that typically deal only with Cohen forcing. In
particular, it allows for notions of forcing that are recursive in the given 
$\overline{\mathbf{GL}_{2}}$ or $\mathbf{ANR}$ degree and so are directly
applicable to results that, for example, involve statements about coding the
given set. It thus applies directly to results about cupping (join)
properties and jump inversion and can be used to simplify the proofs of such
theorems from Jockusch and Posner [1978]. It also includes notions of
forcing whose conditions are objects such as finite trees which are more
complicated than binary strings. The proof we provide is also simpler than
the standard ones (even) for oracle constructions as in Jockusch and Posner
[1978] or Lerman [1983 III.5] in that we eliminate the procedure of, given
the current string $\sigma $ (an initial segment of our eventual generic $G$%
), appointing a string $\tau \supset \sigma $ as a target (to satisfy some
density requirement) and moving toward $\tau $ one step at a time while at
every step checking to see if some target for a higher priority requirement
has been located. While this procedure makes sense for binary strings, it is
hard to see what to make of it in more general settings when the forcing
conditions are more complicated. Instead, we provide a method that, at every
step, attempts to satisfy the highest possible requirement not currently
satisfied. While we cannot always immediately satisfy the next requirement
as in the simpler Kleene-Post finite extension arguments, these attempts
eventually succeed for each requirement. We then present a couple of
illustrative applications for known results (at times extended from $%
\overline{\mathbf{GL}_{2}}$ to $\mathbf{ANR}$) that are proven by ad hoc
arguments in the literature. New results for $\mathbf{ANR}$ degrees
including Theorem \ref{anr} that $\mathbf{ANR}$ degrees are $\mathbf{RRE}$
are presented in \S \ref{anrsec} . This supplies the result needed to unify
those of Shore [1981]\emph{\ } and [2007] as described above.

In \S \ref{ngl2sec}, we provide direct proofs of weaker natural variants, as
well as the full result, of Theorem \ref{asdwy} for $\overline{\mathbf{GL}%
_{2}}$ degrees. To be more precise, we note that the proof of Theorem \ref%
{asdwy} in [ASDWY, Theorem 1.5], while very ingenious and clever, is quite
indirect and nonuniform. It proceeds by first establishing that any $\mathbf{%
a}\leq \mathbf{0}^{\prime }$ which is not $\mathbf{L}_{2}$ (i.e. $\mathbf{a}%
^{\prime \prime }>\mathbf{0}^{\prime \prime }$) is $\mathbf{RRE}$. This
argument relies on results of Harizanov [1998] to convert the problem into
one of finding an infinite ascending or descending chain in a linear order
constructed from $\mathbf{a}$ and then on (a modification of) one of
Hirschfeldt and Shore [2007] to (nonuniformly) produce such a chain that is
low and even 1-generic. [ASDWY] then use a modification of a result of
Jockusch and Posner [1978] for $\overline{\mathbf{GL}_{2}}$ degrees (proved
below for $\mathbf{ANR}$ as Theorem \ref{jp}) and one of Yu [2006] as well
as the Robinson Jump Interpolation Theorem [1971] and another result of
Jockusch and Posner [1978] to reduce the general case to a relativization of
the one for degrees below $\mathbf{0}^{\prime }$ not in $\mathbf{L}_{2}$. In
contrast, our proof that even $\mathbf{ANR}$ degrees are $\mathbf{RRE}$ is
uniform in a witness that the given degree is $\mathbf{ANR}$ as defined and
explained in Definition \ref{anrreldef}, Proposition \ref{anrrelprop} and
Theorem \ref{anr}. In Theorem \ref{ngl21gen} we extend a more elaborate
coding strategy introduced in Definition \ref{wo} that provides a 1-generic $%
\mathbf{c}$ in which the given $\overline{\mathbf{GL}_{2}}$ degree is $%
\mathbf{RRE}$. We close our treatment of $\overline{\mathbf{GL}_{2}}$
degrees by using a notion of forcing in which conditions are finite trees to
give a direct proof (Theorem \ref{ngl2tree}) of the full version of Theorem %
\ref{asdwy} that is uniform relative to the choice of a function witnessing
the specific instance of the degree being in $\overline{\mathbf{GL}_{2}}$
required by the construction.\footnote{%
Anwering Question 4.2 of [ASDWY], Wang [2012] has now shown that every $%
\mathbf{RRE}$ degree is r.e. in a 1-generic below it. In contrast to our
proofs here, however, his are not uniform.}

Quite surprisingly, Cai [2012] shows the property that every $\mathbf{b\geq a%
}$ is $\mathbf{RRE}$ characterizes the $\mathbf{ANR}$ degrees and answers
Questions 4.3 and 4.4 of [ASDWY]:

\begin{theorem}
\emph{(Cai [2012]) }A degree $\mathbf{a}$ is $\mathbf{ANR}$ if and only if
every $\mathbf{b\geq a}$ is $\mathbf{RRE}$.
\end{theorem}

\section{Domination and Forcing $\mathbf{\label{domsec}}$}

In this section we begin by analyzing the definition of array
nonrecursiveness. We have in mind two goals. One (motivated in part by
Theorem \ref{anrrrerel} and Question \ref{iter}) is to develop the correct
relativized version. The other is to strengthen the known domination
properties for these functions and degrees. The strengthenings will make the
notion seem more similar to the domination characterization of $\overline{%
\mathbf{GL}_{2}}$ degrees. They will also provide a stronger general theorem
about meeting dense sets to construct generic sequences for a larger class
of collections of dense sets than had been previously handled. Indeed, they
will provide a common framework for the construction of generic sequences
below both $\mathbf{ANR}$ and $\overline{\mathbf{GL}_{2}}$ degrees.

Recall that the basic domination theoretic definition of array
nonrecursiveness as given originally in [DJS] is for degrees:

\begin{definition}
\label{anrdeg}A degree $\mathbf{a}$ is $\mathbf{ANR}$ if for every function $%
g\leq _{wtt}0^{\prime }$ there is an $f\leq _{T}\mathbf{a}$ such that $f$%
\emph{\ is not dominated by} $g$, i.e. $\exists ^{\infty }x(f(x)>g(x))$.
\end{definition}

What then should be the correct relativized definition that $\mathbf{a}$ is
array nonrecursive relative to $\mathbf{b}$? We should at least have $%
\mathbf{b\leq a}$. One might first try also requiring that for some (or
perhaps any) $B\in \mathbf{b}$ and any $g\leq _{wtt}B^{\prime }$ there is an 
$f\leq _{T}\mathbf{a}$ not dominated by $g$. This, however, would not be
sufficient to relativize the usual results about $\mathbf{ANR}$ degrees to
the realm above $\mathbf{b}$. (In fact, it is not hard to see that there is
a single function recursive in $0^{\prime }$ which dominates every $g$ wtt
below any set $X$.) Another possibility might be to include all functions $f$
computable from $B^{\prime }$ with use bounded by a function recursive in $A$%
. This too is insufficient. One needs to allow unbounded access to $B$.
Along these lines a stronger version would be that for any $g$ computable
from $B\oplus B^{\prime }$ such that the use from $B^{\prime }$ is bounded
by a function recursive in $B$ there is an $f\leq _{T}\mathbf{a}$ not
dominated by $g$. Other variations also seem plausible. A simpler route is
provided by an alternate characterization of $\mathbf{ANR}$ degrees from
[DJS] that depends (in the unrelativized case) on only a single function $g$%
, the modulus of $0^{\prime }$:

\begin{notation}
We let $m$ be the least modulus function of $0^{\prime }$, i.e. $m(x)$ is
the least $s\geq x$ such that $0_{s}^{\prime }\upharpoonright x=0^{\prime
}\upharpoonright x$ where $0_{s}^{\prime }$ is the standard enumeration of $%
0^{\prime }$. Note that $m$ is nondecreasing. Similarly we let $m_{h}$ (or $%
m_{A}$) be the least modulus function for the standard enumeration of $%
h^{\prime }$ ($A^{\prime }$) relative to $h$ ($A$). (We view sets as
represented by their characteristic functions.)
\end{notation}

\begin{proposition}
([DJS]) \label{mod}A degree $\mathbf{a}$ is $\mathbf{ANR}$ if and only if
there is a function $f\leq _{T}\mathbf{a}$ which is not dominated by $m$.
\end{proposition}

We propose to turn this Proposition into a definition which relativizes in
an obvious way.

\begin{definition}
\label{anrreldef}A function $f$ is $ANR$ if it is not dominated by $m$. It
is $ANR$ relative to $h$ if $h\leq _{T}f$ and $f$ is not dominated by $m_{h}$%
. A degree $\mathbf{a}$ is $\mathbf{ANR}$ relative to $\mathbf{b}$, $\mathbf{%
ANR(b)}$, if there are $f\in \mathbf{a}$ and $h\in \mathbf{b}$ such that $f$
is $ANR$ relative to $h$, $ANR(h)$.
\end{definition}

Note that when $\mathbf{b=0}$ this definition agrees with the standard one
for $\mathbf{a}$ being $\mathbf{ANR}$ by Proposition \ref{mod}. (One also
needs to note that, in general, if there is a $g\leq _{T}X$ not dominated by 
$h$ then there is an $f\equiv _{T}X$ which is also not dominated by $h$.
Simply take $f(n)=2g(n)+X(n)$.)

We now provide a domination property that characterizes being $\mathbf{ANR}(%
\mathbf{h})$ but is stronger than the ones previously presented in the
literature even in the unrelativized case. It also shows that our (seemingly
weak) definition in the relativized case is stronger than all the other
possible definitions mentioned after Definition \ref{anrdeg}. It also makes $%
\mathbf{ANR}$ seem much more similar to $\overline{\mathbf{GL}_{2}}$ than
did previous definitions. Recall that $\mathbf{a\in \overline{\mathbf{GL}_{2}%
}}$ if for every function $g\leq _{T}\mathbf{a\vee 0}^{\prime }$, there is
an $f\leq _{T}\mathbf{a}$ which is not dominated by $g$. Our proposition
similarly says that if $f$ is $ANR$ and $g=\Theta (f\oplus 0^{\prime })$
with $0^{\prime }$ use bounded by a function $r\leq _{T}f$ (not just a
recursive function) then there is a $k\leq _{T}f$ which is not dominated by $%
g$. We now state and prove the relativized version by substituting an
arbitrary $h^{\prime }$ for $0^{\prime }$ and also make the uniformities
explicit.

\begin{proposition}
\label{anrrelprop}If $f$ is $ANR(h)$ and $g=\Theta (f\oplus h^{\prime })$
with the $h^{\prime }$ use bounded by a function $r\leq _{T}f$ then there is
a $k\leq _{T}f$ which is not dominated by $g$. Moreover $k$ can be found
uniformly in $f$ in the sense that there is a recursive function $s(e,i,j)$
such that if $\Theta =\Phi _{e}$, $r=\Phi _{i}(f)$ and $h=\Phi _{j}(f)$ then 
$\Phi _{s(e,i,j)}(f)$ will serve as the required $k$.
\end{proposition}

\begin{proof}
Without loss of generality or uniformity we may assume that $f$, $g$ and $r$
are increasing. We define the required $k\leq _{T}f$ as follows: To compute $%
k(n)$ compute, for each $m>n$ in turn, $\Theta _{fr(m)}(f\oplus (h^{\prime
})_{fr(m)};n)$ (i.e. compute $fr(m)$ many steps in the standard enumeration
of $h^{\prime }$ from $h$ and then, using this set as the second component
of the oracle (and $f$ for the first), compute $\Theta $ at $n$ for $fr(m)$
many steps) until the computation converges and then add $1$ to get the
value of $k(n)$. (This procedure must converge as $\Theta (f\oplus h^{\prime
};n)$ converges.) Now as $m_{h}$ does not dominate $f$, there are infinitely
many $n$ such that there is a $j\in \lbrack r(n),r(n+1))$ with $%
f(j)>m_{h}(j) $. For such $n$ we have $fr(n+1)>f(j)>m_{h}(j)\geq m_{h}r(n)$.
Thus $(h^{\prime })_{fr(m)}\upharpoonright r(n)=(h^{\prime })\upharpoonright
r(n)$ for every $m>n$. So the computation of $\Theta (f\oplus h^{\prime };n)$
is, step by step, the same as that of $\Theta (f\oplus (h^{\prime
})_{fr(m)};n)$ for each $m>n$ as all the oracles agree on the actual use of
the true computation. So eventually we get an $m>n$ such that $\Theta
_{fr(m)}(f\oplus (h^{\prime })_{fr(m)};n)\downarrow $ and the output must be 
$\Theta (f\oplus h^{\prime };n)$. Thus for these $n$, $k(n)=g(n)+1>g(n)$ as
required. The uniformity of the definition of $k$ from $f$ and the functions
of the hypotheses is clear. (Noting that we can uniformly, in $f$ and the
reduction of $h$ to $f$, compute the standard enumeration of $h^{\prime }$
from $h$.)
\end{proof}

\begin{corollary}
A degree $\mathbf{a}$ is $\mathbf{ANR(b)}$ if and only if for every $h\in 
\mathbf{b}$ there is a $k\in \mathbf{a}$ such that $k$ is $ANR(h)$.
\end{corollary}

\begin{proof}
The only if direction is immediate from Definition \ref{anrreldef}. The
other direction follows easily from the Proposition since given one $h\in 
\mathbf{b}$ with $f\in \mathbf{a}$ which is $ANR(h)$, the modulus function
of any $\hat{h}\equiv _{T}h$ is given by a function of the type specified in
the hypotheses of the Proposition and so the function $k\leq _{T}f$ provided
by the Proposition is not dominated by $m_{\hat{h}}$ and as noted above we
may as well take $k\equiv _{T}f$. It is then the required $ANR(\hat{h})$
function.
\end{proof}

Iterating the notion of relative array recursiveness also provides some
interesting questions. (A degree $\mathbf{a}$ is array recursive relative to 
$\mathbf{b}$ if it is not $\mathbf{ANR(b)}$.)

\begin{question}
\label{iter}If $\mathbf{0=a}_{0}\mathbf{<a}_{1}<\mathbf{a}_{2}<\cdots <%
\mathbf{a}_{n}$ is a sequence of degrees such that $\mathbf{a}_{i+1}$ is
array recursive relative to $\mathbf{a}_{i}$ for each $i$, what can be said
about $\mathbf{a}_{n}$? If, for example, one could prove that $\mathbf{a}%
_{n}\in \mathbf{GL}_{2}$ then one could show that no $\overline{\mathbf{GL}%
_{2}}$ degree is the top of a finite maximal chain of degrees and so answer
this question from Lerman [1983, p. 87] who shows that the top of any such
maximal chain below $\mathbf{0}^{\prime }$ must be in $\mathbf{L}_{2}$%
\footnote{%
Lerman's question has now been answered by Cai [2011] who has shown that
there can be such a sequence of length $2$ with $\mathbf{a}_{2}\in \mathbf{GH%
}_{2}$.}. Showing that $\mathbf{a}_{n}\notin \mathbf{GH}_{1}$ would provide
interesting information.
\end{question}

We want to consider notions of forcing $\mathcal{P}$ in which the underlying
set $P$ of conditions (thought of as a subset of $\mathbb{N}$) and the
extension relation $\leq _{\mathcal{P}}$ are both recursive in a specified
set $A$. We call such notions of forcing $A$-recursive (or $\mathbf{a}$\emph{%
-recursive }when $A\in \mathbf{a}$). Rather than using $\mathcal{C}$\emph{-}%
generic filters (for a class $\mathcal{C}=\{D_{n}|n\in \mathbb{N\}}$) of
dense sets) we work with $\mathcal{C}$\emph{-}generic sequences $%
\left\langle p_{i}\right\rangle $ of conditions: $\forall i(p_{i}\leq _{%
\mathcal{P}}p_{i+1})$ and $\forall n\exists i(p_{i}\in D_{n})$ and a \emph{%
density function} $d(p,n)$ such that $\forall p\in P\forall n(d(p,n)\leq _{%
\mathcal{P}}p~\&~d(p,n)\in D_{n})$. (We actually construct the generic
sequences we need using such functions. Going from the sequence to the
filter it generates is not always a recursive operation.)

The basic fact about degrees $\mathbf{a\in \overline{\mathbf{GL}_{2}}}$
being able to construct generic sequences (for Cohen forcing) is Lemma 6 of
Jockusch and Posner [1978] stating that if $\mathcal{C}=\left\langle
D_{n}\right\rangle $ is a sequence of dense sets (in $2^{<\omega }$)
uniformly recursive in $A\oplus 0^{\prime }$ for any $A\in \mathbf{a}$ then
there is a $\mathcal{C}$-generic sequence recursive in $\mathbf{a}$. (In
fact, as [DJS] pointed out, it is easy to see that this condition also
implies (and so is equivalent to) $\mathbf{a\in \overline{\mathbf{GL}_{2}}}$%
.) We wish to generalize this result to arbitrary notions of forcing that
are $\mathbf{a}$-recursive. We give a construction more direct than the
original and usual one in that at each step we move (if at all) directly to
the condition that seems to get into the first $D_{n}$ that our sequence
does not yet seem to have met rather searching for a \textquotedblleft best
possible\textquotedblright\ target then moving towards it step by step and
perhaps changing our mind before reaching it. Also note that the idea of
moving toward a condition $p$ step by step that makes natural sense when the
conditions are binary sequences does not make any obvious sense when they
are arbitrary numbers under an arbitrary order relation.

We also give an argument that works (under the appropriate assumptions and
with minor variations) for both $\overline{\mathbf{GL}_{2}}$ and $\mathbf{ANR%
}$ degrees. For $\mathbf{a\in \overline{\mathbf{GL}_{2}}}$ the natural
formulation of the necessary condition on the sequence $\left\langle
D_{n}\right\rangle $ of dense sets that we want to meet is that it is
uniformly recursive in $A\oplus 0^{\prime }$. What we actually use in the
construction is a density function. In this setting, the existence of the
desired function $d$ always follows from, and is usually equivalent to, the
density of the $D_{n}$ and their being uniformly recursive in $A\oplus
0^{\prime }$. This is no longer the case when we move from $\overline{%
\mathbf{GL}_{2}}$ to $\mathbf{ANR}$ and so from Turing reducibility to wtt
reducibility. (For example, one cannot get the required $d\leq _{wtt}A\oplus
0^{\prime }$ from the assumption that the $D_{n}$ are dense and uniformly
wtt reducible to $A\oplus 0^{\prime }$ as its definition requires an
unbounded search.) Thus to handle $\mathbf{ANR}$ degrees we would naturally
turn to density functions as is done for Cohen forcing in [DJS]. To make the
proofs in the two cases as similar as possible, we use them for the $%
\overline{\mathbf{GL}_{2}}$ case as well. Note that, by Proposition \ref%
{anrrelprop},we can actually get by with a weaker hypothesis in the $\mathbf{%
ANR}$ case than might be expected that is closer to that for $\overline{%
\mathbf{GL}_{2}}$. For notational convenience we state and prove the
unrelativized versions of the theorem but, given the definitions and results
above, relativization (to $\overline{\mathbf{GL}_{2}}(\mathbf{b)}$ and $%
\mathbf{ANR(b)}$) is routine.

\begin{theorem}
\label{dom}Suppose $\mathcal{P}$ is an $A$-recursive notion of forcing, $%
\mathcal{C}=\left\langle D_{n}\right\rangle $ a sequence of sets dense in $%
\mathcal{P}$ with a density function $d(x,y)=\Psi (A\oplus 0^{\prime };x,y)$.

\begin{enumerate}
\item[(i)] If $A\in \overline{\mathbf{GL}_{2}}$ then there is a $\mathcal{C}$%
-generic sequence recursive in $A$.

\item[(ii)] If $A\in \mathbf{ANR}$ and the use from $0^{\prime }$ in the
computation of $\Psi (A\oplus 0^{\prime };x,y)$ is bounded by a function $%
\hat{r}\leq _{T}A$, then there is also a $\mathcal{C}$-generic sequence
recursive in $A$.
\end{enumerate}

In both cases the sequence $\left\langle p_{s}\right\rangle $ constructed is 
$\mathcal{C}$-generic because $\forall n\exists s(p_{s+1}=d(p_{s},n))$.
Moreover, in the $\mathbf{ANR}$ case the generic sequence is uniformly
computable from any $ANR$ $f\in \mathbf{a}$ (as a function of the indices of 
$\Psi $ and of $\hat{r}$ relative to $f$).
\end{theorem}

\begin{proof}
Let $\hat{r}(x,y)$ be a function that bounds the $0^{\prime }$ use in the
computation of $\Psi (A\oplus 0^{\prime };x,y)$. Without loss of generality
we may assume that $\hat{r}(x,y)$ is increasing in both $x$ and $y$. In case
(i) we may clearly take $\hat{r}\leq _{T}A\oplus 0^{\prime }$ and in case
(ii) we may take $\hat{r}\leq _{T}A$ by hypothesis. Next note that the
nondecreasing function $m\hat{r}(s,s)$ in case (i) is recursive in $A\oplus
0^{\prime }$ and in case (ii) satisfies the hypotheses of Proposition \ref%
{anrrelprop}, i.e. it is computable from $A\oplus 0^{\prime }$ and its $%
0^{\prime }$ use is bounded by a function ($\hat{r}(s,s)$) recursive in $A$.
Finally note that the maximum of the running times of $\Psi (A\oplus
0^{\prime };x,y)$ for $x,y\leq s$ is also is such a function in each case.
(We run $\Psi $ on each input and then output the sum of the number of steps
needed to converge.) Finally, we let $r$ be the maximum of these three
functions so it too is of the desired form. We now have, by the basic
characterization of $\overline{\mathbf{GL}_{2}}$ degrees or Proposition \ref%
{anrrelprop}, an increasing function $g\leq _{T}A$ not dominated by $r$. We
use $g$ to construct the desired generic sequence $p_{s}$ by recursion.

We begin with $p_{1}=\mathbf{1}$. At step $s+1$ we have (by induction) a
nested sequence $\left\langle p_{i}|i\leq s\right\rangle $ with $p_{i}\leq s$%
. We calculate $0_{g(s+1)}^{\prime }$ and see if there are any changes on
the use from $0^{\prime }$ in a computation based on which some $D_{m}$ was
previous declared satisfied. If so, we now declare it unsatisfied. Suppose $%
n $ is the least $m<s+1$ such that $D_{m}$ is not now declared satisfied.
(There must be one as we declare at most one $m$ to be satisfied at every
stage and none at stage $1$.) We compute $\Psi _{g(s+1)}(A\oplus
0_{g(s+1)}^{\prime };p_{s},n)$. If the computation does not converge or
gives an output $q$ such that $q>s+1$ or $q\nleq _{\mathcal{P}}p_{s}$ we end
the stage and set $p_{s+1}=p_{s}$. Otherwise, we end the stage, declare $%
D_{n}$ to be satisfied on the basis of this computation of the output $q$
and set $p_{s+1}=q$.

We now verify that the sequence constructed is $\mathcal{C}$-generic and
indeed $\forall n\exists s(p_{s+1}=d(p_{s},n))$. Clearly if we ever declare $%
D_{n}$ to be satisfied (and define $p_{s+1}$ accordingly) and it never
becomes unsatisfied again then $p_{s+1}=d(p_{s},n)$. Moreover, if we ever
declare $D_{n}$ to be satisfied (and define $p_{s+1}$ accordingly) and it
remains satisfied at a point of the construction at which we have enumerated 
$0^{\prime }$ correctly up to $r(p_{s},n)$, then by definition $%
p_{s+1}=d(p_{s},n)$ and $D_{n}$ is never declared unsatisfied again. We now
show that this happens.

Suppose all $D_{m}$ for $m<n$ have been declared satisfied by $s_{0}$ and
are never declared unsatisfied again. Let $s+1\geq s_{0}$ be least such that 
$g(s+1)\geq r(s+1)$. If $D_{n}$ was declared satisfied at some $t+1\leq s$
on the basis of some computation of $\Psi _{g(t+1)}(A\oplus
0_{g(t+1)}^{\prime };p_{t},n)$ and there is no change in $0^{\prime }$ on
the use of this computation by stage $g(s+1)$ then the computation is
correct, $p_{t+1}=\Psi (A\oplus 0^{\prime };p_{t},n)\in D_{n}$ and $D_{n}$
is never declared unsatisfied again. (The point here is that by our choice
of $s$, $g(s+1)>m\hat{r}(s+1,s+1)\geq m\hat{r}(p_{t},n)$ and so $%
0_{g(s)}^{\prime }\upharpoonright \hat{r}(p_{t},n)=0^{\prime
}\upharpoonright \hat{r}(p_{t},n)$.) Otherwise, $D_{n}$ is unsatisfied at $s$
and the least such. By construction we compute $\Psi _{g(s+1)}(A\oplus
0_{g(s+1)}^{\prime };p_{s},n)$. The definition of $r$ along with our choice
of $g$ and $s$ guarantee that this computation converges and is correct and
so unless $d(p_{s},n)>s+1$ we declare $D_{n}$ satisfied, set $%
p_{s+1}=d(p_{s},n)$ and $D_{n}$ is never declared unsatisfied again. If $%
d(p_{s},n)>s+1$, we set $p_{s+1}=p_{s}$ and, as $D_{n}$ remains unsatisfied
and the computations already found do not change, we continue to do this
until we reach a stage $v+1\geq d(p_{s},n)$ at which point $p_{v}=p_{s}$ and
we set $p_{v+1}=d(p_{v},n)$ declare $D_{n}$ satisfied and it is never
unsatisfied again.

The uniformity required in the $\mathbf{ANR}$ case is immediate from
Proposition \ref{anrrelprop} and our construction.
\end{proof}

The uniformity provided in the $\mathbf{ANR}$ case of this Theorem carries
over to most constructions of degrees recursive in a given $\mathbf{ANR}$
one. We describe them explicitly in a number of results below. One classic
example is the result of DJS that every $\mathbf{ANR}$ degree bounds a
1-generic degree. Our construction shows that there is a single $e$ such
that, for every $ANR$ function $f$, $\Phi _{e}(f)$ is 1-generic (see also
Proposition \ref{anr1gen}).

\section{$\mathbf{ANR}$ degrees are $\mathbf{RRE\label{anrsec}}$}

In this section we give a number of applications of the basic Theorem \ref%
{dom} for $\mathbf{ANR}$ degrees including the result that they are $\mathbf{%
RRE}$. We begin by extending a theorem of Jockusch and Posner [1978] from $%
\overline{\mathbf{GL}_{2}}$ to $\mathbf{ANR}$. Even for the $\overline{%
\mathbf{GL}_{2}}$ case, it does not fall under the usual paradigm since it
makes demands on coding that require a notion of forcing that is $\mathbf{a}$%
-recursive but not recursive.

\begin{theorem}
\label{jp}If $\mathbf{a}\in \mathbf{ANR}$ and $\mathbf{c}\geq \mathbf{a}\vee 
\mathbf{0}^{\prime }$ and is r.e.\ in $\mathbf{a}$, then there is a $\mathbf{%
g}\leq \mathbf{a}$ s.t. $\mathbf{g}^{\prime }=\mathbf{c}$.
\end{theorem}

\begin{proof}
First fix an $A\in \mathbf{a}$ and an $A$-recursive enumeration $\langle
C_{s}\rangle $ of $C$. The conditions in our notion of forcing $\mathcal{P}$
are binary strings $\sigma $ but extension is defined to reflect the given
enumeration of $C$. We let $F(\sigma )=\{e|\Phi _{e}^{\sigma}(e)\downarrow
\} $. (We employ the usual conventions so that, for example, the computation
of $\Phi _{e}^{\tau }(x)$ requires at least $x$ many steps to converge and
runs only for $|\tau |$ many steps so $F$ is a recursive function and its
values are finite sets.)

If $\tau \supseteq \sigma $ (and so $F(\tau )\supseteq F(\sigma )$) and for
any $e\leq \min (\{|\sigma |\}\cup (F(\tau )-F(\sigma )))$, and for any $%
\langle e,s\rangle \in \lbrack |\sigma |,|\tau |)$, $\tau (\langle
e,s\rangle )=C_{|\sigma |}(e)$ we say that $\tau \leq _{\mathcal{P}}\sigma $%
. We make this into the required partial order by taking the transitive
closure of this relation. The transitive closure is also recursive in $A$
because we can lay out and check the finitely many possible one step paths
between any $p$ and $q$ to see if any of them satisfy the original relation
at each step. The intuition (as in the Shoenfield [1959] jump theorem) is
that, whenever we try to extend a string, we want to make sure that some
(eventually growing) initial segment of columns respects the enumeration of $%
C$ in sense that $\tau (\langle e,s\rangle )=C_{|\sigma |}(e)$ and so $%
\lim_{s\rightarrow \infty }G(\left\langle e,s\right\rangle =C(e)$. This
makes $C\leq _{T}G^{\prime }$ by the Shoenfield limit lemma.

Define our sequence $\mathcal{C}$ of sets as follows: 
\begin{equation*}
D_{n,j}=\{\sigma :|\sigma |>j~\&\ [\Phi _{n}^{\sigma }(n)\downarrow 
\hbox{
or }\forall \tau \supset \sigma \lbrack \Phi _{n}^{\tau }(n)\uparrow 
\hbox{
or }
\end{equation*}%
\begin{equation*}
\exists s\exists e<\min (\{|\sigma |\}\cup (F(\tau )-F(\sigma )))(\langle
e,s\rangle \in \lbrack |\sigma |,|\tau |)\ \&\ \tau (\langle e,s\rangle
)\neq C_{|\sigma |}(e))]]\}\text{.}
\end{equation*}%
We calculate the required density function $d(\sigma ,\left\langle
n,j\right\rangle )$ for the $D_{n,j}$ as follows: Given any $\sigma ,n$ and $%
j$ we may as well assume (by, recursively in $A$, taking a long enough
extension $\rho $ with $\rho (\left\langle e,s\right\rangle )=C_{|\sigma
|}(e)$ for every $\left\langle e,s\right\rangle >|\sigma |,\left\langle
e,s\right\rangle \leq j$ ) that $|\sigma |>j$. Now check whether $\Phi
_{n}^{\sigma }(n)\downarrow $, if so, set $d(\sigma \left\langle
n,j\right\rangle )=$ $\sigma \in D_{n,j}$ and we are done. If not, use $A$
to get all the values of $C_{|\sigma |}(e)$ for $e\leq |\sigma |$. Ask ($%
0^{\prime }$) whether we can find an extension $\tau $ of $\sigma $ with the
property that for all $e\leq \min (\{|\sigma |\}\cup (F(\tau )-F(\sigma )))$
and all $s$ such that $\langle e,s\rangle \in \lbrack |\sigma |,|\tau |)$,
we have $\tau (\langle e,s\rangle )=C_{|\sigma |}(e)$, and $\Phi _{n}^{\tau
}(n)\downarrow $. If so, we let $d(\sigma ,\left\langle n,j\right\rangle )$
be the first such $\tau $ (found in a standard ordering of computations). It
is immediate that $d(\sigma ,\left\langle n,j\right\rangle )\leq _{\mathcal{P%
}}\sigma $ and $\tau \in D_{n,j}$. Otherwise, we let $d(\sigma ,\left\langle
n,j\right\rangle )=\sigma \in D_{n,j}$.

As we determined $C_{|\sigma |}$ recursively in $A$, the $0^{\prime }$ use
for the question asked is clearly bounded by a function recursive in $A.$
Thus, by Theorem \ref{dom}(ii), we have a $\mathcal{C}$-generic sequence $%
\langle \sigma _{i}\rangle $ recursive in $A$. We let $G=\cup \{\sigma
_{p_{i}}\}\leq _{T}A$.

First, we claim that $C\leq _{T}G^{\prime }$ and, in particular, $C(n)=\lim
G(n,s)$ for every $n$. Given $n$, there is obviously a $j$ such that for
every $e\leq n$, $e\in G^{\prime }\Leftrightarrow \Phi _{e}^{\sigma
_{j}}(e)\downarrow $ and $C_{|\sigma _{j}|}(e)=C(e)$. By the definition of
our forcing notion, $G(n,t)=C_{|\sigma _{j}|}(n)=C(n)$ for $t\geq |\sigma
_{j}|$.

To see that $G^{\prime }\leq _{T}C$, assume we have determined $G^{\prime
}\upharpoonright n$ and want to decide if $n\in G^{\prime }$. Recursively in 
$C\geq _{T}A\vee 0^{\prime }$ find $j$ and $k$ large enough so that $%
C\upharpoonright n=C_{j}\upharpoonright n$, $\sigma _{k+1}=d(\sigma
_{k},\left\langle n,j\right\rangle )$ and $G^{\prime }\upharpoonright
n=F(\sigma _{k+1})\upharpoonright n$. (It is clear, first, that there are
such $j$ and $k$ and then that they can be recognized recursively in $C$
which computes both the sequence $\sigma _{i}$ and $d$.) It is now clear
from the definition of $D_{n,j}$ and our notion of forcing that $n\in
G^{\prime }\Leftrightarrow \Phi _{n}^{\sigma _{k}}(n)\downarrow $.
\end{proof}

We now apply our general theorem to an $A$-recursive notion of forcing
chosen to produce relative recursive enumerability.

\begin{theorem}
\label{anr} If $\mathbf{a}\in \mathbf{ANR}$ then $\mathbf{a}$ is $\mathbf{RRE%
}$. Indeed, there are $e$ and $i$ such that, for every $ANR$ function $f$, $%
\Phi _{e}(f)<_{T}f$ and $W_{i}^{\Phi _{e}(f)}\equiv _{T}f$.
\end{theorem}

\begin{proof}
Suppose $f\in \mathbf{a}$ is $ANR$. Let $A$ be the graph of $f$ (and so $%
A\equiv _{T}f$). We use an $A$-recursive notion of forcing $\mathcal{P}$
with conditions $p=\langle p_{0},p_{1},p_{2}\rangle $, $p_{i}\in 2^{<\omega
} $ such that

\begin{enumerate}
\item $|p_{0}|=|p_{1}|$, $p_{0}(d_{n})=A(n-1)$, $p_{1}(d_{n})=1-A(n-1)$
where $d_{n}$ is $n^{th}$ place where $p_{0},p_{1}$ differ and

\item $(\forall e<|p_{0}\oplus p_{1}|)(e\in p_{0}\oplus p_{1}\Leftrightarrow
\exists x(\langle e,x\rangle \in p_{2}))$.
\end{enumerate}

\noindent Extension in this notion of forcing is defined simply by $q\leq _{%
\mathcal{P}}p\Leftrightarrow q_{i}\supseteq p_{i}$ but note that this
applies only to $p$ and $q$ in $P$. Membership in $P$ and $\leq _{\mathcal{P}%
}$ are clearly recursive in $A$.

Our plan is to define a class $\mathcal{C}$ of dense sets $D_{n}$ with a
density function $d(p,n)$ recursive in $A\oplus 0^{\prime }$ with $0^{\prime
}$ use recursively bounded. Theorem \ref{dom}(ii) then supplies a $\mathcal{C%
}$-generic sequence $\left\langle p_{s}\right\rangle \leq _{T}A$ from which
we can define the required $G\leq _{T}A$ in which $\mathbf{a}$ is r.e.\ If $%
p_{s}=\left\langle p_{s,0},p_{s,1},p_{s,2}\right\rangle $ we let $G_{i}=\cup
\{p_{s,i}|s\in \mathbb{N\}}$ for $i=0,1,2$ so $G_{i}\leq _{T}A$. {Then, if
we can force $G_{0}$ and $G_{1}$ to differ at infinitely many places, $%
G_{0}\oplus G_{1}\equiv _{T}A$.} On the other hand, the definition of the
notion of forcing obviously makes {$G_{0}\oplus G_{1}$ r.e.\ in }$G_{2}$.
Thus $\mathbf{a}$ will be r.e.\ in $\mathbf{g=}\deg (G_{2})$. We will have
other requirements that make $\mathbf{g<a}$ as well.

We begin with the dense sets that provide the differences we need: 
\begin{equation*}
D_{2n}=\{p\in \mathcal{P}:p_{0},p_{1}\text{ differ at at least $n$ points}\}.
\end{equation*}%
We define the required function $d(r,2n)$ by recursion on $n$. Given $r$ and 
$n+1$, we suppose we have calculated $d(r,2n)=p=\langle
p_{0},p_{1},p_{2}\rangle \in D_{2n}$ with $p\leq _{\mathcal{P}}r$. If $%
p\notin D_{2n+2}$, we need to compute a $q=\langle q_{0},q_{1},q_{2}\rangle
\in D_{2n+2}$ with $q\leq _{\mathcal{P}}p$. Let $q_{0}=p_{0}\symbol{94}A(n)$%
, $q_{1}=p_{1}\symbol{94}(1-A(n))$. Choose $i\in \{0,1\}$ such that $%
q_{i}(|p_{0}|)=1$. Define $q_{2}\supseteq p_{2}$ by choosing $x$ large and
setting $q_{2}(\left\langle 2|p_{0}|+i,x\right\rangle )=1$ and $q_{2}(z)=0$
for all $z\notin \limfunc{dom}(p_{2})$ and less than $\left\langle
2|p_{0}|+i,x\right\rangle $. Now $q=\langle q_{0},q_{1},q_{2}\rangle $
satisfies the requirements to be a condition in $P$. It obviously extends $p$
and is in $D_{2n+2}$. This computation is clearly recursive in $A$.

We must now add dense sets to guarantee that $A\nleq _{T}G_{2}$. A direct
route is to let 
\begin{eqnarray}
D_{2n+1} &=&\{p\in \mathcal{P}:~\exists x(\Phi _{n}^{p_{2}}(x)\downarrow
\neq A(x))~\text{or }\forall (\sigma _{0},\sigma _{1}\supseteq
p_{2})[\exists x(\Phi _{n}^{\sigma _{0}}(x)\downarrow \neq \Phi _{n}^{\sigma
_{1}}(x))\downarrow \Rightarrow   \notag \\
(\exists i &\in &\{0,1\})(\exists \left\langle e,x\right\rangle
)(e<|p_{0}\oplus p_{1}|~\&~\sigma _{i}(\left\langle e,x\right\rangle )=1\neq
(p_{0}\oplus p_{1})(e)]\}.  \notag
\end{eqnarray}%
Of course, the first alternative guarantees that $\Phi _{n}^{G_{2}}\neq A$
while the second that $\Phi _{n}^{G_{2}}$, if total, is recursive. The point
here is that if some $p_{s}$ in our generic sequence satisfies the second
clause then, we can, for any $z$, calculate $\Phi _{n}^{G_{2}}(z)$ by
finding any $\sigma \supseteq p_{s,2}$ such that $\Phi _{n}^{\sigma
}(z)\downarrow $ and taking its value as $\Phi _{n}^{G_{2}}(z)$. There is
such a $\sigma \subseteq G_{2}$ as $\Phi _{n}^{G_{2}}$ is assumed to be
total and $G_{2}\supseteq p_{s,2}$. If there were some other $\tau \supseteq
p_{s,2}$ with $\Phi _{n}^{\tau }(z)\downarrow \neq \Phi _{n}^{\sigma
}(z)\downarrow $ then, by our choice of $s$ and the definition of $D_{2n+1}$%
, there is no $\left\langle e,x\right\rangle $ with $e<|p_{0}\oplus p_{1}|$
such that $\tau (\left\langle e,x\right\rangle )=1\neq (p_{0}\oplus p_{1})(e)
$. Thus we could form a condition $q\leq _{\mathcal{P}}p_{s}$ with $%
q_{2}=\tau $ by extending $p_{0}$ and $p_{1}$ by setting $q_{1}(w)=q_{2}(w)=1
$ (for $w\geq |p_{0}|$) if either $\left\langle 2w,v\right\rangle $ or $%
\left\langle 2w+1,v\right\rangle $ is in $\tau $ for any $v$. In this way,
no new differences between $q_{0}$ and $q_{1}$ (not already in $p_{0}$ and $%
p_{1}$) occur and the definition of being a condition is satisfied. Thus $q$
is a condition extending $p_{s,2}$ with $\Phi _{n}^{q_{2}}(z)\downarrow \neq
A(z)$ contradicting our choice of $s$.

We compute the required density function $d(q,2n+1)$ as follows. Given $q$
we ask one question of $0^{\prime }$ determined recursively in $q$: Are
there extensions $\sigma _{0},\sigma _{1}$ of $q_{2}$ that would show that $%
q $ does not satisfy the second disjunct in the definition of $D_{2n+1}$. If
not, let $d(q,2n+1)=q$ which is already in $D_{2n+1}$. If so, we find the
first such pair (appearing in a recursive search) and ask $A$ which $\sigma
_{i}$ gives an answer different from $A(x)$. We now need a condition $%
r=d(q,2n+1)$ extending $q$ with third coordinate $r_{2}$ extending $\sigma
_{i}$. For each $\left\langle e,x\right\rangle $ with $e\geq |q_{1}\oplus
q_{2}|)$ and $\sigma _{i}(\left\langle e,x\right\rangle )=1$ we define $%
r_{j}(z)=1$ for both $j\in \{0,1\}$ for the $z$ that makes $(r_{0}\oplus
r_{1})(e)=1$ and otherwise we let $r_{j}(u)=0$ for all other $u$ less than
the largest element put into either $r_{0}$ or $r_{1}$ by the previous
procedure. We now extend $\sigma _{i}$ to the desired $r_{2}$ by putting in $%
\left\langle k,y\right\rangle $ for a large $y$ for all those $k\geq |q_{1}|$
put into $r_{0}\oplus r_{1}$ for which there is no $\left\langle
k,w\right\rangle $ in $\sigma _{i}$. Otherwise we extend $\sigma _{i}$ by $0$
up to the largest element put in by this procedure. It is clear that this
produces a condition $r$ as required. (No points of difference between $%
r_{0} $ and $r_{1}$ are created that were not already present in $q$.)

We now apply Theorem \ref{dom} to get a $\mathcal{C}$-generic sequence $%
\left\langle p_{s}\right\rangle \leq _{T}A$. As promised, we let $G_{i}=\cup
\{p_{s,i}|s\in \mathbb{N\}}$ for $i=0,1,2$ and, as described above, $A\equiv
_{T}G_{0}\oplus G_{1}$ which is r.e.\ in $G_{2}$. In addition, the
conditions in $D_{2n+1}$ guarantee (as above) that $\Phi _{n}^{G_{2}}\neq A$
as well. The uniformity assertions follow immediately from those in Theorem %
\ref{dom} and our construction.
\end{proof}

We now point out some additional information about $G_{2}$ that can be
extracted from this construction.

\begin{proposition}
\label{relngl2}For the $G_{2}$ constructed in the proof of Theorem \ref{anr}
such that the given $A\in \mathbf{ANR}$ is r.e.\ in and strictly above $%
G_{2} $ we also have $G_{2}^{\prime }\equiv _{T}A\oplus 0^{\prime }$ and so
if $A\in \overline{\mathbf{GL}_{2}}$ then $A$ is also $\overline{\mathbf{GL}%
_{2}}(G_{2})$, i.e. $(A\oplus G_{2}^{^{\prime }})^{\prime }<_{T}A^{\prime
\prime } $.
\end{proposition}

\begin{proof}
We first claim that $G_{2}^{\prime }\leq _{T}A\vee 0^{\prime }$. To see if $%
e\in G_{2}^{\prime }$, recursively compute an $n$ such that, for every $\tau 
$, $\Phi _{n}^{\tau }(x)=\tau (x)$ if $\Phi _{e}^{\tau }(e)\downarrow $ and
is divergent otherwise. Now, recursively in $A\oplus 0^{\prime }$ find an $s$
such that $p_{s+1}=d(p_{s},2n+1)$. If $p_{s+1}$ is in $D_{2n+1}$ because of
the first clause of the definition then $\Phi _{n}^{p_{s+1,2}}(x)\downarrow $
for some $x$ and so $e\in G_{2}^{\prime }$. Otherwise we claim $e\notin
G_{2}^{^{\prime }}$. Suppose for the sake of a contradiction that, for some $%
t$, $\Phi _{e}^{p_{t,2}}(e)\downarrow $. It is now easy to get extensions of 
$p_{t,2}$ that show that $p_{s+1}$ does not satisfy the second clause of the
definition of $D_{n}$ for the desired contradiction. Simply choose $%
y>2|p_{s,0}|,|p_{t,2}|$ and extend $p_{t,2}$ by $0$ up to $\left\langle
y,0\right\rangle $ and then with $i=0,1$ at $\left\langle y,0\right\rangle $
to get the required $\sigma _{i}$. On the other hand, as $\mathbf{a}$ is
r.e.\ in $G_{2}$, $A\oplus 0^{\prime }\leq _{T}G_{2}^{\prime }$ and $%
G_{2}^{\prime }\equiv _{T}A\oplus 0^{\prime }$ as desired. If $\mathbf{a}\in 
\overline{\mathbf{GL}_{2}}$, $\mathbf{a}^{\prime \prime }>(\mathbf{a\vee 0}%
^{\prime })^{\prime }$ and so $\mathbf{a}^{\prime \prime }>(\mathbf{a\vee g}%
_{2}^{\prime })^{\prime }$ as required.
\end{proof}

A reasonable question now would be to ask for an analogous result for $%
\mathbf{ANR}$ degrees to that given in this Proposition for $\overline{%
\mathbf{GL}_{2}}$ based on our Definition \ref{anrreldef} of relative array
nonrecursiveness.

\begin{theorem}
\label{anrrrerel}If $\mathbf{a}$ is $\mathbf{ANR}$ then there is a $\mathbf{g%
}$ relative to which $\mathbf{a}$ is both r.e.\ and $\mathbf{ANR}$.
\end{theorem}

\begin{proof}
We replace the sets $D_{2n+1}$ of the proof of Theorem \ref{anr} with new
ones (also called $D_{2n+1}$) that force a maximal number of convergences of 
$\Phi _{n}^{G_{2}}(m)$. We here directly specify the (calculation procedure
for the) associated density functions $d(p,2n+1)$: We ask $2^{p}$ many
questions of $0^{\prime }$. For each subset $F$ of $\{i|i<p\}$ we ask if
there is a $\sigma \supseteq p_{2}$ \textquotedblleft adding no new
numbers\textquotedblright\ (i.e. $\lnot \exists \left\langle
e,x\right\rangle (e<|p_{0}\oplus p_{1}|~\&~\sigma (\left\langle
e,x\right\rangle )=1\neq (p_{0}\oplus p_{2})(\left\langle e,x\right\rangle )$%
) that makes $\Phi _{n}^{\sigma }(m)\downarrow $ for every $m\in F$. We take
a maximal such $F$ and find the first extension $\sigma >p$ witnessing the
convergences for $m\in F$. We then get an extension $q$ of $p$ with third
coordinate extending $\sigma $ as before. Note that by the usual coding of
binary sequences and triples, $q>p$ as well. By induction then if $g(m)$ is
the $m$th stage $s$ at which we have $p_{s+1}=d(p_{s},2n+1)$ for some $n$, $%
p_{g(m)}>m$. Note that this procedure also satisfies the hypotheses of
Theorem \ref{dom}(ii). The $D_{2n+1}$ are declared satisfied and unsatisfied
as before but note that they become unsatisfied when we discover that the $F$
associated with the extension used was not maximal (this is, after all, part
of the computation on which we based the calculation of $d$). By the proof
of Theorem \ref{dom}, there are infinitely many $m$ such that the $D_{2n+1}$
declared satisfied at $g(m)$ is never declared unsatisfied. So in particular
for such $m$, for every $e<m$ with $e\in G_{2}^{\prime }$, $\Phi
_{e}^{p_{g(m)+1,2}}(e)\downarrow $. Thus if we now define $h(m)$ as the
stage in the standard enumeration of $G_{2}^{\prime }$ at which all the
numbers $e$ such that $\Phi _{e}^{p_{g(m)+1,2}}(e)\downarrow $ have been
enumerated in $G_{2}^{\prime },$ then, for each of these infinitely many $m$%
, $h(m)$ will be at least as large as the modulus function for $%
G_{2}^{\prime }$ (relative to $G_{2}$) at $m$.
\end{proof}

In the next section we will improve Theorem \ref{anr} when $\mathbf{a\in 
\overline{\mathbf{GL}_{2}}}$ by making the degree witnessing that $\mathbf{a}
$ is $\mathbf{RRE}$ 1-generic. Here we present another extension to $\mathbf{%
ANR}$ of a result result from [ASDWY] about $\overline{\mathbf{GL}_{2}}$
both as an illustration of our general methodology as well as an
introduction of the coding techniques that will be exploited in the next
section.

\begin{definition}
\label{wo}For any set $G$, we define a relationship $\mathbf{as}(e,G)=\sigma 
$, the number $e$ is \emph{assigned to (the string)} $\sigma $ to hold when $%
\sigma $ is the shortest $\tau \subset G$ such that $\Phi _{e}^{\tau
}(e)\downarrow $. If $\mathbf{as}(e,G)=\sigma $, we define the \emph{weak
order} \emph{of} $e$ \emph{along} $G$, $\mathbf{wo}(e,G)$, recursively as
one more than the weak order of $e^{\prime }$ along $G$ where $e^{\prime }$
is the unique $x<e$ such, for some $\tau \subset \sigma $, $\mathbf{as}%
(x,G)=\tau $ and for no $e^{\prime \prime }<x$ and $\rho $ with $\tau
\subset \rho \subset \sigma $ is $\mathbf{as}(e^{\prime \prime },G)=\rho $.
If there is no such $x$, then $\mathbf{wo}(e,G)=0$. We define $\mathbf{as}%
(e,\rho )=\sigma $ and the weak order of $e$ along $\rho $ similarly for
strings $\rho $.
\end{definition}

Intuitively, we first assign numbers to strings along $G$, and then for any $%
e$ assigned to $\sigma $, we search downward from $\sigma $ for the first $%
e^{\prime }<e$ assigned. The weak order of $e$ is then the weak order of $%
e^{\prime }$ plus $1$. We use these notions to code $\mathbf{a}$ into $%
\mathbf{g}^{\prime }$ in Proposition \ref{anr1gen}.

\begin{remark}
Without loss of generality we may clearly choose our master list of
computations so that for any $\tau $ there is at most one $e$ such that $%
\Phi _{e}^{\tau }(e)\downarrow $ but $\Phi _{e}^{\tau ^{-}}(e)\uparrow $. So
along any $G$ each node is assigned at most one number.
\end{remark}

\begin{notation}
For any string $\sigma $, we let $\sigma ^{-}$ be the initial segment of $%
\sigma $ gotten by removing the last number in $\sigma $. If there is a set $%
G$ (string $\tau $) that is clear from the context such that $\sigma \subset
G$ ($\tau $), $\sigma ^{+}$ will denote $\sigma \symbol{94}G(|\sigma |)$ ($%
\sigma \symbol{94}\tau (|\sigma |)$).
\end{notation}

\begin{proposition}
\label{anr1gen}If $\mathbf{a}\in \mathbf{ANR}$ then there is an $1$-generic
degree $\mathbf{g}$ recursive in $\mathbf{a}$ such that $\mathbf{g}^{\prime
}=\mathbf{a}\vee \mathbf{0}^{\prime }$. Indeed, there is an $e$ such that,
if $f$ is $ANR$, then $\Phi _{e}^{f}$ is 1-generic and $\Phi _{e}(f)^{\prime
}\equiv _{T}f\oplus 0^{\prime }$ (and this equivalence is also given
uniformly).
\end{proposition}

\begin{proof}
We again begin with $A$ being the graph of $f$. Our forcing conditions are
binary strings. Membership of $\rho $ in $\mathcal{P}$ is defined
recursively: if $\sigma \subsetneq \rho $ is the longest initial segment of $%
\rho $ such that $\mathbf{as}(e,\rho )=\sigma $ for some $e$ then we require
that $\sigma ^{+}(|\sigma |)=A(\mathbf{wo}(e,\rho ))$; moreover, if $\tau $
is the longest initial segment of $\sigma $ such that $\mathbf{as}(e^{\prime
},\rho )=\tau $ for some $e^{\prime }<e$ then $\tau ^{+}$ is also
(recursively required to be) in $\mathcal{P}$. (By default, for the base
case, if no number is assigned to any $\sigma \subset $ $\tau $, then $\tau
\in \mathcal{P}$.) Thus $\mathcal{P}$ is clearly recursive in $A$. The order 
$\sigma \leq _{\mathcal{P}}\tau $ for our notion of forcing is the usual
extension relation on strings, $\sigma \supseteq \tau $.

Our dense sets will be:%
\begin{equation*}
D_{n}=\{\sigma \in \mathcal{P}|\Phi _{n}^{\sigma }(n)\downarrow ~\text{or~}%
(\forall \tau \supset \sigma )(\Phi _{n}^{\tau ^{-}}(n)\downarrow ~\text{or~}%
\Phi _{n}^{\tau }(n)\uparrow ~\text{or~}(\forall i)(\tau \symbol{94}i\notin 
\mathcal{P}))\}\text{.}
\end{equation*}%
Now we define an appropriate density function $d(p,n)$. Consider any $\rho
\in \mathcal{P}$. (If $\rho \notin \mathcal{P}$, we let $d(\rho ,n)=1$ for
any $n$.) In order to find $\sigma \leq _{\mathcal{P}}\rho $ in $D_{n}$, we
first check whether $\Phi _{n}^{\rho }(n)\downarrow $, if so we are done, if
not then ask whether there is $\sigma \supset \rho $ such that $\Phi
_{n}^{\sigma }(n)\downarrow $, $\Phi _{n}^{\sigma ^{-}}(n)\uparrow $ and $%
\sigma \symbol{94}A(\mathbf{wo}(n,\sigma ))\in \mathcal{P}$. If so, $\sigma 
\symbol{94}A(\mathbf{wo}(n,q))\in D_{n}$ and $\sigma \symbol{94}A(\mathbf{wo}%
(n,q)\leq _{\mathcal{P}}\rho .$ If not, $\rho \in D_{n}$. The second
question only requires $A\upharpoonright n$ to determine the question it
asks of $0^{\prime }$. The rest of the procedure is recursive in $A$ and so $%
d$ satisfies the hypotheses of Theorem \ref{dom}(ii).

Theorem \ref{dom}(ii) now provides a sequence $\left\langle \sigma
_{i}\right\rangle \leq _{T}A$ meeting all the $D_{n}$. Our desired $G$ is $%
\cup \sigma _{i}$. So $G\leq _{T}A$. To see that $G$ is $1$-generic, we need
to check that for any $e$ there is a node $\sigma \subset G$ which forces $%
e\in G^{\prime }$ or $e\notin G^{\prime }$. Note that there is an $i$ such
that $\sigma _{i}$ decides if $e^{\prime }\in G^{\prime }$ for every $%
e^{\prime }<e$ (by induction) and $\sigma _{i+1}\in D_{e}$ (by the
genericity of the sequence and the closure of the $D_{n}$ under extension in 
$\mathcal{P}$). Of course, if $\Phi _{e}^{\sigma _{i+1}}(e)\downarrow $ we
are done. Otherwise, we claim that $\sigma _{i+1}$ forces $e\notin G^{\prime
}$. If not, then we can find the first $\tau \supset \sigma _{i}$ such that $%
\Phi _{e}^{\tau }(e)\downarrow $, take a one-bit extension of $\tau $ by
coding in $A(\mathbf{wo}(e,\tau ))$, then this $\tau ^{+}\in \mathcal{P}$
because all $e^{\prime }<e$ in $G^{\prime }$ are forced in by strings
shorter than $\sigma _{i}$. This contradicts the fact that $\sigma _{i+1}\in
D_{n}$.

Now we have $G^{\prime }\leq _{T}G\vee 0^{\prime }\leq _{T}A\vee 0^{\prime }$
and it remains to show that $A\leq _{T}G^{\prime }$. We say some pair $%
(e,\tau )$ has true weak order $n=\mathbf{wo}(e,\tau )$, if no string
extending $\tau $ can have an assignment $e^{\prime }$ with weak order $n$,
i.e., any $e^{\prime }$ assigned after $\tau $ is greater than $e$. For such 
$a$ pair $(e,\tau )$, we must have $\tau ^{+}(|\tau |)=A(\mathbf{wo}(e,\tau
))=A(n)$. Finally, for every $n$ we can use $G^{\prime }$ to find the unique
pair $(e,\tau )$ with true weak order $n$, so $A\leq _{T}G^{\prime }$.

As usual the desired uniformities are immediate from those of Theorem \ref%
{dom}(ii) and our construction.
\end{proof}

\section{$\overline{\mathbf{GL}_{2}}$ degrees\label{ngl2sec}}

Theorem \ref{anr} provides a procedure that, given any $\mathbf{ANR}$ degree 
$\mathbf{a}$, produces a $\mathbf{g}<\mathbf{a}$ in which $\mathbf{a}$ is
r.e. We now show how to make $\mathbf{g}$ 1-generic if $\mathbf{a\in 
\overline{\mathbf{GL}_{2}}}$ much more directly than is done in [ASDWY].

\begin{theorem}
\label{ngl21gen}\emph{([ASDWY]) }If $\mathbf{a}\in \overline{\mathbf{GL}_{2}}
$, then it is r.e.\ in and above a $1$-generic degree.
\end{theorem}

\begin{proof}
Given a finite string $\sigma$, we define the \emph{rank} of $\sigma$ ($%
\mathbf{rk}(\sigma)$) as the maximum weak order along $\sigma$.

Define a notion of forcing $\mathcal{P}$ consisting of all $p=\langle
p_{0},p_{1},p_{2}\rangle $, $p_{0},p_{1},p_{2}\in 2^{<\omega }$ such that:

\begin{enumerate}
\item $p_{0}$ and $p_{1}$ are of the same length and if $m$ is the $n$th
position at which they differ, $p_{0}(m)=1-p_{1}(m)=A(n-1)$.

\item For $n<|p_{0}\oplus p_{1}|$, $n\in p_{0}\oplus p_{1}$ if and only if
there exists $\sigma \subsetneq p_{2}$ and an $e$ assigned to $\sigma $ with
weak order $n$ and $p_{2}(|\sigma |)=1$ (i.e. $\sigma ^{+}(|\sigma |)=1$).

\item $2|p_{0}|=\mathbf{rk}(p_{2})$.
\end{enumerate}

Let $q\leq _{\mathcal{P}}p$ if $q_{i}\supset p_{i}$ for each $i$. Clearly
this notion of forcing is $A$-recursive. Intuitively, if we get $\langle
A_{0},A_{1},G\rangle $ from a sufficiently generic sequence (recursive in $A$%
), then $A\equiv _{T}A_{0}\oplus A_{1}$ which is r.e.\ in (and above) $G$.
To determine whether $n\in A_{0}\oplus A_{1}$, one simply runs through $G$
checking whether there is a node with weak order $n$ and a $1$ coded right
after that. We also want to guarantee the $1$-genericity of $G$.

Given $p\in \mathcal{P}$, we say $\tau \supset p_{2}$ \emph{respects} $p$ if
there is no $(n,\sigma )$ such that $n<|p_{0}\oplus p_{1}|$, $n\notin
p_{0}\oplus p_{1}$, $\sigma $ is between $p_{2}$ and $\tau $ with some
number $e$ assigned to it and of weak order $n$ and $\tau (|\sigma |)=1$. So
if $\tau $ does not respect $p$, then we cannot extend $p$ to a condition $q$
with $q_{2}$ extending $\tau $ because any such extension would violate
clause (2) in the definition of our notion of forcing. Conversely, the
following holds:

\begin{lemma}
\label{l1}If $p\in \mathcal{P}$, $\mathbf{rk}(p_{2})>n$, $\Phi
_{n}^{p_{2}}(n)\uparrow $, $\tau \supset p_{2}$ respects $p$ and $n$ is
assigned to $\tau $, then there exists a $q\leq _{\mathcal{P}}p$ such that $%
q_{2}\supset \tau $ and so one can be found recursively in $A$.
\end{lemma}

\begin{proof}
We first try to extend $p_{0}$ and $p_{1}$. If we use $\tau $ and blindly
follow the dictates of clause (2) of our definition of $\mathcal{P}$ to
extend $p_{0},p_{1}$, we might violate clause (1). However, it is easy to
see that we can fix this by extending $\tau $ by putting $1$'s after some
nodes with certain weak orders between $\mathbf{rk}(p_{2})$ and $\mathbf{rk}%
(\tau )$. Notice that $\mathbf{wo}(n,\tau )\leq n$ and $\mathbf{rk}(p_{2})>n$%
, and those weak orders that need adjustments are $>n$, so we can extend $%
\tau $ and wait for those weak orders to appear, then code in $1$'s after
some of them so that we meet the requirements of clause (1).
\end{proof}

Define dense sets:%
\begin{equation*}
D_{n}=\{p:\Phi _{n}^{p_{2}}(n)\downarrow \hbox{ or }\forall \tau \supset
p_{2}(\Phi _{n}^{\tau }(n)\uparrow )\}.
\end{equation*}%
These $D_{n}$ will guarantee that $G=\cup _{i}\{p_{i,2}\}$ is $1$-generic.
We need to define a density function recursively in $A\oplus 0^{\prime }$.

Given $p$, assume $\Phi _{n}^{p_{2}}(n)\uparrow $ and $\mathbf{rk}(p_{2})>n$
(as we can always extend a string respectfully). We first ask whether there
is a $\tau $ extending $p_{2}$ which makes $\Phi _{n}^{\tau }(n)\downarrow $
and respects $p$. If so we find the first such $\tau $ and notice that $n$
is assigned to $\tau $. Then, by Lemma \ref{l1}, we can find $q\in D_{n}$
extending $p$. If not, we claim that we can find $q\leq _{\mathcal{P}}p$
with $q\in D_{n}$ and $\Phi _{n}^{q_{2}}(n)\uparrow $, i.e. $q$ forces $%
n\notin G^{\prime }$.

Next we ask ($0^{\prime }$) whether there is a $\tau $ extending $p_{2}$
which makes $\Phi _{n}^{\tau }(n)\downarrow $. If not then $p$ is already in 
$D_{n}$. If so, find the first one $\tau _{1}$, then, by the negative answer
to our first question, we know that $\tau _{1}$ does not respect $p$. Find
the first initial segment $\eta _{1}$ of $\tau $ which does not respect $p$,
i.e., $\eta _{1}^{-}$ is assigned a number with weak order $<\mathbf{rk}%
(p_{2})$, $\eta _{1}(|\eta _{1}|-1)=1$ but the definition of $\eta _{1}$
respecting $\tau $ would require it to be $0$. Then put $\xi _{1}=\eta
_{1}^{-}\ast 0$, that is, we change the last bit to respect $p$.

Note that $\Phi _{n}^{\xi _{1}}(n)\uparrow $ because $\xi _{1}$ respects $p$%
. Now ask ($0^{\prime }$) whether $\xi _{1}$ forces $n\notin G^{\prime }$.
If so, then we are done by (a slight variation of) Lemma \ref{l1}. If not,
we repeat this process: find $\tau _{2}\supset \xi _{1}$ which makes $\Phi
_{n}^{\tau _{2}}(n)\downarrow $, then find the first initial segment of $%
\tau _{2}$ which does not respect $p$, change the last bit and get a $\xi
_{2}$ which respects $p$.

We can continue this process but at each repetition we need some new
extension assigned a number with weak order $<\mathbf{rk}(p_{2})$. Therefore
this process cannot continue forever, i.e. we will eventually stop and get a 
$\xi _{i}$ which respects $p$ and which forces $n\notin G^{\prime }$.
Finally use Lemma \ref{l1} to get $q\leq _{\mathcal{P}}p$, $q_{2}\supset \xi
_{i}$ and $q\in D_{n}$.

To make sure that $A_{0}$ and $A_{1}$ have infinitely many points of
difference we add another sequence of dense sets:%
\begin{equation*}
D_{n}^{\ast }=\{p:p_{0},p_{1}\hbox{ differ at at least $n$ positions}\}
\end{equation*}%
This is similar to the last part of Lemma \ref{l1}: for any $p$, one has to
be careful extending $p_{2}$ while still satisfying clauses (1) and (2) and
yet adding a point of difference. Since we don't have any other
requirements, this process is easy and recursive in $A$.
\end{proof}

Our final step is to prove that if $\mathbf{a\in }$\textbf{$\overline{%
\mathbf{GL}_{2}}$} then every $\mathbf{b\geq }~\mathbf{a}$ is r.e.\ in a
1-generic strictly below it. We prove a seemingly quite different
proposition from which this result will easily follow.

\begin{proposition}
Every $\mathbf{a}\in \overline{\mathbf{GL}_{2}}$ computes an infinite binary
tree $T$ in $2^{<\omega }$ such that for any path $C\in \lbrack T]$, $C$ is $%
1$-generic, $C$ does not compute $\mathbf{a}$ and $\mathbf{a}$ is r.e.\ in $%
C $ (but not necessarily above $C$).
\end{proposition}

\begin{proof}
We now also arrange our master list of computations so that there is no $%
\tau $ of even length such that $\Phi _{e}^{\tau }(e)\downarrow $ but $\Phi
_{e}^{\tau ^{-}}(e)\uparrow $. Thus no string of even length is assigned a
number.

We say a string $\tau $ is $A$-admissible if there exist $p_{0},p_{1}$ s.t. $%
\langle p_{0},p_{1},\tau \rangle $ is a forcing condition in the sense of
the previous construction. Note that $p_{0}$ and $p_{1}$ are uniquely
determined by $\tau $. We will denote them by $p_{0}(\tau )$ and $p_{1}(\tau
)$, respectively.

We say $\tau$ respects $\sigma$ if $\tau$ respects $\langle
p_0(\sigma),p_1(\sigma),\sigma\rangle$ as in the previous construction.

Now we define a new notion of forcing: $\mathcal{P}$ consists of finite
binary trees such that every leaf is $A$-admissible. We let $q\leq _{%
\mathcal{P}}p$ if $q$ is a binary tree extending $p$, and each leaf of $q$
respects the corresponding leaf in $p$ that it extends.

We define dense sets:%
\begin{equation*}
D_{n}=\{p:\hbox{ for every leaf $\sigma$ of p, }\Phi _{n}^{\sigma
}(n)\downarrow \hbox{ or }\forall \tau \supset \sigma (\Phi _{n}^{\tau
}(n)\uparrow )\}.
\end{equation*}%
\begin{equation*}
D_{n}^{\ast }=\{p:%
\hbox{ for every leaf $\sigma$ of p, $p_0(\sigma)$ and
$p_1(\sigma)$ differ at at least $n$ positions}\}
\end{equation*}%
These two types of dense sets are handled in the same way as in the previous
construction. For every leaf, after we find an extension satisfying the
conditions in $D_{n}$ (or $D_{n}^{\ast }$), we can assume that it has even
length and extend it by $0$ and $1$ to split it into two leaves. This
preserves $A$-admissibility since no number is assigned to nodes of even
length.

Next, we want to make sure that no path $C$ can compute $A$. Define
additional dense sets as follows:%
\begin{equation*}
E_n=\{p: \hbox{ for every leaf $\sigma$ of $p$, } [\exists
x\Phi_n^\sigma(x)\downarrow\neq A(x)
\end{equation*}%
\begin{equation*}
\hbox{ or } \exists x\forall\tau\supset\sigma(%
\hbox{$\tau$ respects
$\sigma$}\Rightarrow \Phi_n^\tau(x)\uparrow)]\}
\end{equation*}

Now given $\sigma $, a leaf of $p$, we first fix a recursive list of
strictly increasing indices $n_{0}<n_{1}<...<n_{i}<...$ such that if $\Phi
_{n}^{\tau }(i)\downarrow $ then for any $\tau ^{\prime }\supset \tau $
which is large enough to allow for the spacing required by the conditions we
imposed on our master list of computations, $\Phi _{n_{i}}^{\tau ^{\prime
}}(n_{i})\downarrow $, and conversely $\Phi _{n_{i}}^{\tau ^{\prime
}}(n_{i})\uparrow $ if no $\tau \subset \tau ^{\prime }$ makes $\Phi
_{n}^{\tau }(i)\downarrow $.

Let $\sigma _{i}=\sigma \ast 0^{j}$ where $j$ is the least such that $%
\mathbf{rk}(\sigma \ast 0^{j})>n_{i}$ and $\mathbf{rk}(\sigma \ast 0^{j})$
is even. Using $0^{\prime }$ and $A$ we go through the $\sigma _{i}$ asking
whether: 
\begin{equation*}
\forall \tau \supset \sigma _{i}(\hbox{$\tau$ respects $\sigma_i$}%
\Rightarrow \Phi _{n}^{\tau }(i)\uparrow )
\end{equation*}

If we ever get a\textquotedblleft yes\textquotedblright\ answer for some $%
\sigma _{i}$, we output this $\sigma _{i}$ (note that $\sigma _{i}$ is
always $A$-admissible and respects $\sigma $). If we get a \textquotedblleft
no\textquotedblright\ answer for $\sigma _{i}$, we then find the first such $%
\tau _{i}\supset \sigma _{i}$ which respects $\sigma _{i}$ and which makes $%
\Phi _{n}^{\tau _{i}}(i)\downarrow $. If $\Phi _{n}^{\tau _{i}}(i)=$ $A(i)$
we proceed to $i+1$. If $\Phi _{n}^{\tau _{i}}(i)\neq $ $A(i)$, then extend $%
\tau _{i}$ to $\eta _{i}=\tau _{i}\ast 0^{k}$ for the first $k$ such that $%
\eta _{i}$ is assigned the number $n_{i}$. Now this $\eta _{i}$ respects $%
\sigma _{i}$ and by Lemma \ref{l1} we can find an extension $\eta $ of $\eta
_{i}$ which is $A$-admissible, and then output $\eta $.

Now we prove that we always halt in this process: Suppose not, then for any $%
\sigma_i$ we would always get a ``yes'' answer and could find the first $%
\tau_i\supset\sigma_i$ which respects $\sigma_i$ and $\Phi_n^{%
\tau_i}(i)=A(i) $. That would make $A$ recursive.

Finally we get extensions of all leaves of $p$ and then branch each them
into two in the same way as in our analysis of $D_{n}$ and $D_{n}^{\ast }$.
Now $E_{n}$ forces that, for each path $C$, either $\Phi _{n}^{C}$ is not
total, or it is not $A$.
\end{proof}

\begin{theorem}
\label{ngl2tree}If $\mathbf{a}\in \overline{\mathbf{GL}_{2}}$ and $\mathbf{b}%
\geq \mathbf{a}$, then $\mathbf{b}$ is r.e.\ in and strictly above a $1$%
-generic $\mathbf{c}$. Moreover, a $C\in \mathbf{c}$ can be found uniformly
effectively in any $B\in \mathbf{b}$ from an index for an $A\in \mathbf{a}$
as a set recursive in $B$ and an index for a function (recursive in $A$ and
hence $B$) not dominated by a particular effectively determined function
recursive $A\oplus 0^{\prime }$.
\end{theorem}

\begin{proof}
Let $T$ be the tree recursive in $A$ constructed in the above Proposition.
Given $B\geq _{T}A$, we let $C$ be the path in the tree gotten by following $%
B$, i.e., $C=T(B)$. It is easy to see that $B\equiv _{T}A\oplus C$, so $B$
is r.e.\ in and above $C$ which is, of course, 1-generic. Moreover, since $%
A\nleq _{T}C$, $C$ is strictly below $B$.

As for the uniformity assertions, we explain what we mean by describing the
procedure. We are given $B$ and an index computing $A$ from $B$. From this
information we can effectively find indices (from $A\oplus 0^{\prime }$) for
the density functions for the sets $D_{n}$, $D_{n}^{\ast }$ and $E_{n}$ and
then for the associated function $r$ (from $A\oplus 0^{\prime }$) used in
the proof of Theorem \ref{dom}. The noneffective step is now to produce an
index for the function $g\leq _{T}A$ which is not dominated by $r$. Given
that index for $g$, the rest of the construction in the proof of Theorem \ref%
{dom} proceeds effectively in $A$ and provides the generic sequence $%
\left\langle p_{i}\right\rangle $ for our construction here and an index for
it from $A$. Going from the sequence to the corresponding tree $T$ and then
to the path $C=T(B)$ is then also uniformly effective in $B$.
\end{proof}

\section{Bibliography}

%TCIMACRO{\TeXButton{parindent}{\hspace*{\parindent}}}%
%BeginExpansion
\hspace*{\parindent}%
%EndExpansion
Ambos-Spies, K., Ding, D., Wang, W. and Yu, L. [2009], Bounding non-GL$_{2}$
and R.E.A., \emph{Journal of Symbolic Logic}, to appear.

Cai, M. [2011], A $2$-minimal non-$\mathbf{GL}_{2}$ degree, \emph{Journal of
Mathematical Logic}, to appear.

Cai, M. [2012], Array nonrecursiveness and relative recursive enumerability, 
\emph{Journal of Symbolic Logic}, to appear.

Downey, R.\ G., Jockusch, C.\ G.\ jr.\ and Stob, M.\ [1990], Array
nonrecursive sets and multiple permitting arguments, in \emph{Recursion
Theory Week}, K.\ Ambos-Spies, G.\ H.\ M\"{u}ller and G.\ E.\ Sacks eds.,
Springer-Verlag, Berlin, 141-174.

Harizanov, V. [1998], Turing degrees of certain isomorphic images of
computable relations, \emph{Ann. Pure and Applied Logic} \textbf{93},
103-113.

Hirschfeldt, D. and Shore, R. A. [2007], Combinatorial Principles Weaker
than Ramsey's Theorem for Pairs, \emph{Journal of Symbolic Logic} \textbf{72}%
, 171-206.

Jockusch, C. G. Jr. and Posner, D. B. [1978], Double jumps of minimal
degrees, \emph{Journal of Symbolic Logic} \textbf{43}, 715-724.

Lerman, M.\ [1983], \emph{Degrees of Unsolvability}, Perspectives in
Mathematical Logic, Springer-Verlag, Berlin.

Robinson, R. W. [1971], Jump restricted interpolation in the recursively
enumerable degrees, \emph{Annals of Mathematics (2)} \textbf{93, }586-596.

Shoenfield, J. R. [1959], On degrees of unsolvability, \emph{Annals of
Mathematics (2)} \textbf{69, }644-53.

Shore, R. A. [1981], The theory of the degrees below $0^{\prime }$, \textit{%
Journal of the London Mathematical Society} \textbf{24} (1981), 1-14.

Shore, R. A. [2007], Direct and local definitions of the Turing jump, \emph{%
Journal of Mathematical Logic }\textbf{7}, 229-262.

Wang, W. [2012], Relative enumerability and 1-genericity, \emph{Journal of
Symbolic Logic}, to appear.

Yu, L. [2006], Lowness for genericity, \emph{Archive for Math. Logic }%
\textbf{45}, 233-238.

\end{document}
