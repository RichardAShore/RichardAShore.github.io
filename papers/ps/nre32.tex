%% This document created by Scientific Word (R) Version 3.0
%\input{tcilatex}
%\input{tcilatex}


\documentclass[12pt]{article}
%%%%%%%%%%%%%%%%%%%%%%%%%%%%%%%%%%%%%%%%%%%%%%%%%%%%%%%%%%%%%%%%%%%%%%%%%%%%%%%%%%%%%%%%%%%%%%%%%%%%%%%%%%%%%%%%%%%%%%%%%%%%%%%%%%%%%%%%%%%%%%%%%%%%%%%%%%%%%%%%%%%%%%%%%%%%%%%%%%%%%%%%%%%%%%%%%%%%%%%%%%%%%%%%%%%%%%%%%%%%%%%%%%%%%%%%%%%%%%%%%%%%%%%%%%%%
\usepackage{amsfonts}
\usepackage{amsmath}
\usepackage{amssymb}
\usepackage{graphicx}
\usepackage{amsopn}
\usepackage{amsthm}

\setcounter{MaxMatrixCols}{10}
%TCIDATA{OutputFilter=LATEX.DLL}
%TCIDATA{Version=5.00.0.2606}
%TCIDATA{<META NAME="SaveForMode" CONTENT="1">}
%TCIDATA{BibliographyScheme=Manual}
%TCIDATA{Created=Monday, May 10, 2010 18:28:03}
%TCIDATA{LastRevised=Sunday, July 18, 2010 11:55:36}
%TCIDATA{<META NAME="GraphicsSave" CONTENT="32">}
%TCIDATA{<META NAME="DocumentShell" CONTENT="Standard LaTeX\Standard LaTeX Article(mine)">}
%TCIDATA{Language=American English}
%TCIDATA{CSTFile=LaTexArt(mine).cst}

\setlength{\evensidemargin}{0in}
\setlength{\oddsidemargin}{0in}
\setlength{\textwidth}{6.25in}
\setlength{\textheight}{8.5in}
\setlength{\topmargin}{0in}
\setlength{\headheight}{0in}
\setlength{\headsep}{0in}
\setlength{\itemsep}{0pt}
\renewcommand{\topfraction}{.9}
\renewcommand{\textfraction}{.1}
\setlength{\parskip}{\smallskipamount}
\newtheorem{theorem}{Theorem}[section]
\newtheorem{axiom}[theorem]{Axiom}
\newtheorem{claim}[theorem]{Claim}
\newtheorem{conclusion}[theorem]{Conclusion}
\newtheorem{condition}[theorem]{Condition}
\newtheorem{conjecture}[theorem]{Conjecture}
\newtheorem{corollary}[theorem]{Corollary}
\newtheorem{lemma}[theorem]{Lemma}
\newtheorem{notation}[theorem]{Notation}
\newtheorem{proposition}[theorem]{Proposition}
\theoremstyle{definition}
\newtheorem{definition}[theorem]{Definition}
\newtheorem{example}[theorem]{Example}
\newtheorem{exercise}[theorem]{Exercise}
\newtheorem{problem}[theorem]{Problem}
\newtheorem{question}[theorem]{Question}
\newtheorem{remark}[theorem]{Remark}
\input{tcilatex}

\begin{document}

\title{The n-r.e. degrees: undecidability and $\Sigma _{1}$ substructures}
\author{Mingzhong Cai \addtocounter{footnote}{1}\thanks{%
Partially supported by NSF Grant DMS-0852811.} \\
%EndAName
Department of Mathematics\\
Cornell University\\
Ithaca NY 14853 \and Richard A. Shore\addtocounter{footnote}{-2}\thanks{%
Partially supported by NSF Grants DMS-0554855, DMS-0852811 and John
Templeton Foundation Grant 13408.} \\
%EndAName
Department of Mathematics\\
Cornell University\\
Ithaca NY 14853 \and Theodore A. Slaman\addtocounter{footnote}{1}\thanks{%
Partially supported by NSF Grant DMS-1001551 and by the John Templeton
Foundation.} \\
%EndAName
Department of Mathematics\\
University of Califonia, Berkeley}
\maketitle

\section{Introduction}

Turing reducibility (introduced in Turing [1939]) captures the intuitive
notion of one set $A\subseteq \mathbb{N}$ being computable from another $B$,
We write $A\leq _{T}B$, $A$ is \emph{Turing reducible} to or \emph{%
computable from} $B$ to mean that there is a Turing machine (program) $\Phi $
that can compute $A$ if given access to an \textquotedblleft
oracle\textquotedblright\ for $B$ in the sense that the computing machine is
augmented by a procedure that allows it to ask for any number $n$ it
computes if $n\in B$ and to receive the correct answer. This reducibility
naturally induces a partial order $\leq _{T}$ on the set $\mathcal{D}$ of
equivalence classes (called \emph{Turing degrees} or simply \emph{degrees}) $%
\mathbf{a=\{}B|A\leq _{T}B~\&~B\leq _{T}A\}$. The structure of $\mathcal{D}$
then captures that of relative complexity of computation of sets and
functions (on $\mathbb{N}$). The study of this relation on all sets
(functions), and on many important subclasses of sets has been a major
occupation of recursion (computability) theory ever since its introduction.

In addition to the full structure, $\mathcal{D}$, the most important
substructures studied have been those of the recursively enumerable degrees, 
$\mathcal{R}$, and $\mathcal{D}(\leq \mathbf{0}^{\prime })$, the degrees
below the halting problem, $K=\{e|\Phi _{e}(e)$ converges$\}$ whose degree
is denoted by $\mathbf{0}^{\prime }$. The recursively enumerable sets are
those which can be enumerated (listed) by a recursive (computable) function.
They can also be seen as those sets $A$ for which there is a very simple
approximation procedure, a recursive function $f(x,s)$ to the characteristic
function $A(x)$ of $A$ such that $\forall x(f(x,0)=0~\&~\lim f(x,s)=A(x))$
that changes its mind about membership in $A$ at most once, i.e. there is at
most one $s$ such that $f(x,s)\neq f(x,s+1)$. Shoenfield's Limit Lemma
[1959] says that the sets (or functions) computable from the halting problem 
$0^{\prime }$ are precisely those with some convergent recursive
approximation, i.e. the sets $A$ such that there is a recursive function $%
f(x,s)$ such that $\forall x(f(x,0)=0~\&~\lim f(x,s)=A(x))$. So, while for
each $x$ there are only finitely many changes, the number of such changes
over all $x$ may be unbounded.

In this paper we study a natural hierarchy of intermediate classes of sets
and degrees. The $n$-r.e. sets are those for which there is a recursive
approximation $f(x,s)$ as above for which there are at most $n$ changes of
value at each $x$. The corresponding degree structures are denoted $\mathcal{%
D}_{n}$, the degrees of the $n$-r.e. sets. (So $\mathcal{D}_{1}=\mathcal{R}$
the r.e. degrees.) This hierarchy was introduced by Putnam [1965] and Gold
[1965]. It was extended into the transfinite by Ershov [1968, 1968a, 1970]
who proved that the sets in the transfinite hierarchy he defined are
precisely those computable from $0^{\prime }$.

The early work on degree theories began with the investigation of local
algebraic or order-theoretic properties of the structures. This work
continues in full force to this day. In the past three decades or so, a more
global approach has emerged as well. Here one studies issues such as the
decidability or, more generally, the complexity of the theories of degree
structures as well as related questions about definability in, and possible
automorphisms of, these structures.

For the first couple of decades, a major motivating idea was that (at least
some of) these structures should be simple and characterizable by basic
algebraic properties. Shoenfield's conjecture [1965] would have been such a
complete characterization of $\mathcal{R}$ analogous to that of the
rationals as the countable dense linear order without endpoints. Even after
the conjecture had been refuted by Lachlan [1966] and Yates [1966], Sacks
[1966] still conjectured that the r.e. degrees were decidable. More recent
results have produced a dramatically different prevailing paradigm for $%
\mathcal{D}$, $\mathcal{D}(\leq _{T}\mathbf{0}^{\prime })$ and $\mathcal{R}$
as well as many degree structures for other notions of reducibility. Rather
than seeing the complexity of the structures as an obstacle to
characterization, it suggests that a sufficiently strong proof of complexity
would completely characterize each structure. Instead of expecting the
structures to be decidable and homogeneous with many automorphisms (like the
rationals), one looks to prove that the theories are as complicated as
possible, there are definable degrees and that the structure has few
automorphisms.\ 

Typical results include the following:

\begin{theorem}
$\mathcal{D}$, $\mathcal{D}(\leq _{T}\mathbf{0}^{\prime })$ and $\mathcal{R}$
are each undecidable by Lachlan [1968]; Epstein [1979] and Lerman [1983];
and Harrington and Shelah [1982], respectively.
\end{theorem}

\begin{theorem}
The theories of $\mathcal{D}$, $\mathcal{D}(\leq _{T}\mathbf{0}^{\prime })$
and $\mathcal{R}$ are as complicated as possible, i.e. recursively
isomorphic to true second order arithmetic for $\mathcal{D}$ and to true
first order arithmetic for $\mathcal{D}(\leq _{T}\mathbf{0}^{\prime })$ and $%
\mathcal{R}$ by Simpson [1977]; Shore [1981]; and Harrington and Slaman and
then Slaman and Woodin (both unpublished) (see Nies, Shore and Slaman [1998]
for a proof and stronger results), respectively.
\end{theorem}

\begin{theorem}
All relations invariant under the double jump that are definable in
arithmetic are definable in $\mathcal{D}$, $\mathcal{D}(\leq _{T}\mathbf{0}%
^{\prime })$ and $\mathcal{R}$ where for $\mathcal{D}$ we mean second order
arithmetic and for the others first order by Slaman and Woodin [2001] (see
Slaman [1991] for an announcement and Shore [2007] for a quite different
proof that applies to various substructures of $\mathcal{D}$ as well),
essentially Shore [1988] (but see also Nies, Shore and Slaman [1998, Theorem
3.11 and the remarks following it])and Nies, Shore and Slaman [1998],
respectively. (The converse holds by the definability of these degree
structures in arithmetic.)
\end{theorem}

A survey paper for this area is Shore [2006].

In this paper we take the first steps on this road for the structures $%
\mathcal{D}_{n}$ by proving that they are all undecidable. We conjecture
that our work can be extended along the lines of Nies, Shore and Slaman
[1998] to show that their theories are also all recursively isomorphic to
that of true arithmetic. Perhaps one can even prove definability results as
done there for $\mathcal{R}$. Basic survey papers on the structure of the $%
\mathcal{D}_{n}$ are Arslanov [2009, 2010] and Stephan, Yang and Yu [2009].

Another important theme in the study of these degree structures has been
delimiting the similarities and explicating the differences among them.
While it is relatively easy to distinguish among $\mathcal{D}$, $\mathcal{D}%
(\leq _{T}\mathbf{0}^{\prime })$ and $\mathcal{R}$ in many way the issue
becomes particularly compelling when we turn to the $\mathcal{D}_{n}$. It is
easy to imagine, and was proved early on, that moving from $\mathcal{R}$ to
all sets or even to the unlimited approximations characterizing those below $%
0^{\prime }$ introduces many differences. For the $\mathcal{D}_{n}$,
however, the question is what does the ability to change precisely one more
time buy us in terms of additional degrees, algebraic structure and
complexity.

Of course, the first question is are the $\mathcal{D}_{n}$ actually
distinct. Indeed, there are, for each $n$, $(n+1)$-r.e. degrees which are
not $n$-r.e. ([Cooper [1971] with the stronger result that they can be found
not even $n$-rea in Jockusch and Shore [1984]]). While the one quantifier
theory of all the degree structures from $\mathcal{R}$ to $\mathcal{D}$ are
the same since one can embed all finite (even countable) partial orderings
into $\mathcal{R}$ (and so all the rest as well), there were many early
results establishing elementary differences between $\mathcal{R}$ and the
other $\mathcal{D}_{n}$ with cupping, density and lattice embedding
properties playing the featured role (as in, for example, Arslanov [1985]
Cooper et al. [1991], Downey [1989], respectively). Differences between any
of the other $\mathcal{D}_{n}$, however, seemed hard to find. Downey [1989]
even conjectured that they might all be elementarily equivalent, i.e. all
sentences (in the first order language with $\leq $) true in any $\mathcal{D}%
_{n}$ for $n\geq 2$ is true in all of them. This conjecture was not refuted
until quite recently. Arslanov, Kalimullin and Lempp [2010] provide an
elementary difference between $\mathcal{D}_{2}$ and $\mathcal{D}_{3}$. In
fact, the sentence they exhibit on which the structures differ is at the
smallest possible level: two quantifiers ($\forall \exists $). They
conjecture (as one would now expect) that the $\mathcal{D}_{n}$ are pairwise
not elementarily equivalent. They also conjecture that this level of
difference ($\forall \exists $) is as small as possible in the strong sense
that every $\exists \forall $ sentence true in any $\mathcal{D}_{n}$ is true
in every $\mathcal{D}_{m}$ for $m\geq n$.

Now an $\exists \forall $ sentence is true if there are choices (parameters
substitutable) for the existentially quantified variables such that the
resulting universal sentence is true of these parameters. The strongest way
that their conjecture could be true is for the same parameters to work in
both structures. This view brings to mind a much earlier question raised
about other pairs of our degree structures. Are any $\Sigma _{1}$
substructures of any others. ($\mathcal{M}$ is a $\Sigma _{1}$\emph{\
substructure} of $\mathcal{N}$, $\mathcal{M}\preceq _{1}\mathcal{N}$, if for
any $\Sigma _{1}$ formula $\exists \bar{y}\varphi (\bar{x},\bar{y})$ where $%
\varphi $ is quantifier free and any choice of elements $\bar{a}$ from $%
\mathcal{M}$, $\mathcal{M}\vDash \exists \bar{y}\varphi (\bar{a},\bar{y}%
)\Leftrightarrow \mathcal{N}\vDash \exists \bar{y}\varphi (\bar{a},\bar{y})$%
.)

Slaman ([1983]) proved early on that this fails at the extreme ends: $%
\mathcal{R}\npreceq _{1}\mathcal{D}(\leq \mathbf{0}^{\prime }$) (and so, 
\emph{a fortiori}, $\mathcal{D}_{n}\npreceq _{1}\mathcal{D}(\leq \mathbf{0}%
^{\prime }$) for any $n\geq 1$. Slaman and then others raised the natural
question of whether it could be that $\mathcal{D}_{n}\preceq _{1}\mathcal{D}%
_{m}$ for any $n<m$. Yang and Yu [2006] provided a negative answer for $n=1$
and $m=2$ (and so for any $m\geq 2$). We complete the picture by showing
that $\mathcal{D}_{n}\npreceq _{1}\mathcal{D}_{m}$ for any $n<m$. (We have
just heard that Arslanov and Jamaleev are preparing a different proof for
the case $n=2$.)

Turning now to our proofs, we begin with undecidability. As usual (see for
essentially our situation \S 2 of Nies, Shore and Slaman [1998] or for a
more general model theoretic treatment Hodges [1993, 5.3]), we have a
formula $\varphi _{D}(x,\bar{p})$ which, for each choice of parameters $\bar{%
p}$, defines a subset $D$ of our structure $\mathcal{D}_{n}$ and another
formula $\varphi _{R}(x,y,\bar{p})$ which defines a binary relation $R$ on $%
D $. To prove undecidability it suffices to show that, as the parameters
vary over $\mathcal{D}_{n}$, a sufficiently rich class of structures $(D,R)$
are coded in this way. In our case, we code partial orders. As the (r.e.)
set of theorems of the theory of partial orders is recursively inseparable
from the (r.e.) set of sentences (of the language of partial orders) that
are false in some finite partial order (Taitslin [1962]), it suffices to
code any collection of relations containing all finite partial orders. The
point here is that if $\mathcal{D}_{n}$ were decidable then the set of
sentences true in every partial order coded by $\varphi _{D}$ and $\varphi
_{R}$ as the parameters $\bar{p}$ range over all elements of $\mathcal{D}_{n}
$ would be recursive. Of course, it contains the theorems of the theory of
partial orders and, if we code all finite ones, is disjoint from the set of
sentences with finite counterexamples. As it turns out, it is no more
difficult to prove that one can code all recursive partial orders than all
finite ones. This is what we do explicitly in our proof of Theorem \ref{po}.
We use the basic idea of the domain being maximal degrees $\mathbf{g\leq
_{T}a}$ not joining some other degree $\mathbf{p}$ above $\mathbf{q}$ from
Harrington and Shelah [1982] and build on their work.

\begin{theorem}
\label{po}Given a recursive partial order $(\omega ,\leq _{\ast })$ and an $%
n\geq 1$, there exist uniformly $n$-r.e.\ sets $G_{i}$ for each $i\in \omega 
$, an $n$-r.e.\ set $L$ and r.e.\ sets $P$ and $Q$ such that:

\begin{enumerate}
\item Each $\mathbf{g}_{i}$ is a maximal $n$-r.e. degree below $\mathbf{a}$
such that $\mathbf{q}\nleq \mathbf{g}_{i}\vee \mathbf{p}$ where $%
A=\bigoplus_{i}G_{i}$.

\item $\mathbf{g}_{i}\leq \mathbf{g}_{j}\vee \mathbf{l}$ if and only if $%
i\leq _{\ast }j$.
\end{enumerate}
\end{theorem}

Thus the required formulas $\varphi _{D}$ and $\varphi _{R}$ defining our
domains and order relations have parameters $\mathbf{a}$, $\mathbf{p}$ and $%
\mathbf{q}$. The first says that $\mathbf{x}$ is a maximal degree below $%
\mathbf{a}$ such that $\mathbf{q}\nleq \mathbf{x}\vee \mathbf{p}$. The
second says that $\mathbf{x}\leq \mathbf{y}\vee \mathbf{l}$. so we have the
desired result.

\begin{theorem}
The theories of $\mathcal{D}_{n}$ are undecidable for every $n$.
\end{theorem}

If instead of recursive inseparability, we wanted to rely only on the
undecidability of the theory of partial orders, we should code all partial
orders recursive in $0^{\prime }$ as every sentence which is not a theorem
(of the theory) has a counterexample recursive in $0^{\prime }$ by the
effective version of the completeness theorem.

One can with only minor modifications not affecting the structure of our
proof handle partial orders recursive in $0^{\prime }.$ We precisely
describe the modifications needed in \ref{0' 0''}. With some additional work
and a serious reorganization of the priority tree, one can get all partial
orders recursive in $0^{\prime \prime }$. One puts in a new type of node
which guesses in a $\Delta _{3}$ procedure at each bit of information about
this partial order and bases later work on these guesses. The added
complexity is considerable without much gain for applications. It seems that
one can even get any $\Sigma _{3}$ partial order by a slightly more
complicated procedure. We briefly describe this procedure in \ref{0' 0''} as
well.

It is worth remarking that our proof works for $n=1$ as well as all larger $%
n $. Indeed, it can be significantly simplified for $n=1$ by omitting all
items that consider the possibility that the $G_{i}$ and $W_{i}$ (the list
of $n$-r.e. sets recursive in $A$) are not r.e. This gives a considerably
simplified proof of the undecidability of $\mathcal{R}$ along the lines
suggested in Harrington and Shelah but with a simpler statement using fewer
parameters and a significantly easier construction. We do not believe any
proof even for $\mathcal{R}$ along these lines has been published before.

We next turn to $\Sigma _{1}$ substructures.

\begin{theorem}
$\mathcal{D}_{n}\npreceq _{1}\mathcal{D}_{m}$ for $n<m$.
\end{theorem}

The technical result needed here is the following generalization of Theorem
1.12 in Yang and Yue [2006] who do the case $n=1$:

\begin{theorem}
\label{n/n+1}For any $n\geq 1$, there are r.e. degrees $\mathbf{g},\mathbf{p}%
,\mathbf{q}$, an $n$-r.e.\ degree $\mathbf{a}$ and an $n+1$-r.e.\ degree $%
\mathbf{d}$ such that:

\begin{enumerate}
\item For every $n$-r.e.\ degree $\mathbf{w}\leq \mathbf{a}$, either $%
\mathbf{q}\leq \mathbf{w}\vee \mathbf{p}$, or $\mathbf{w}\leq \mathbf{g}$.

\item $\mathbf{d}\leq\mathbf{a}$, $\mathbf{q}\nleq\mathbf{d}\vee\mathbf{p}$,
and $\mathbf{d}\nleq\mathbf{g}$.
\end{enumerate}
\end{theorem}

This theorem shows directly that $\mathcal{D}_{n}$ is not a $\Sigma _{1}$
elementary substructure of $\mathcal{D}_{n+1}$ in the language with $\vee $
as well as $\leq $: In $\mathcal{D}_{n}$, no $\mathbf{w}$ below $\mathbf{a}$
has the property that $\mathbf{q}\nleq \mathbf{w}\vee \mathbf{p}$ and $%
\mathbf{w}\nleq \mathbf{g}$ while in $\mathcal{R}_{n+1}$, $\mathbf{d}\leq
_{T}\mathbf{a}$ has both properties. We can eliminate $\vee $ by rephrasing
the property of $\mathbf{w}$ as $\exists \mathbf{z(w,p\leq z~\&~q}\nleq 
\mathbf{z)~\&~w}\nleq \mathbf{g}$ which is $\Sigma _{1}$ in just $\leq $ and
so the existence of a $\mathbf{w}$ with this property is true in $\mathcal{D}%
_{n+1}$ (i.e. $\mathbf{d}$) but false in $\mathcal{D}_{n}$. Of course, as $%
\mathbf{d}$ is in $\mathcal{D}_{m}$ for every $m\geq n+1$, $\mathcal{D}%
_{n}\npreceq _{1}\mathcal{D}_{m}$ as well.

Much of the construction and verification is the same for Theorems \ref{po}
and \ref{n/n+1}. We treat the first theorem as primary. In \S \ref{basic}
where we cover basic notions and conventions common to both, we use curly
brackets \{\} to indicate changes (usually alphabetic only at this stage)
for the second theorem. The rest of the paper is divided into two parts, one
for each of the theorems. Each part describes first the requirements (\S \ref%
{Req}, \ref{Req II}), then the priority tree (\S \ref{tree}, \ref{tree II}),
the construction (\S \ref{con}, \ref{con II}) and finally the verifications
that the construction succeeds (\S \ref{ver}, \ref{ver II}). We describe
everything in full detail for the first theorem and then for the second
describe only the changes needed. In our descriptions of the constructions,
material enclosed in square brackets [] is meant to convey intuition or
describe aspects of the construction that will only be verified later. It is
not part of the formal definition of the construction procedures.

As might be expected from the types of requirements, both constructions are $%
0^{\prime \prime \prime }$ arguments even for the case $n=1$. As these
constructions go, however, ours are at the simpler end: the priority tree is
finitely branching, there is no backtracking and only one type of
requirement is injured along the true path. The key idea for carrying the
arguments from the r.e. case ($n=1$) to the $n$-r.e. one ($n>1$) in Theorem %
\ref{po} is what we call \emph{shuffling }(\S \ref{shuf}). Roughly speaking,
at the crucial $0^{\prime \prime \prime }$ determined nodes, we are
attempting to construct functionals $\Delta $ that, to working towards the
maximality of the $\mathbf{g}_{i}$, try to compute some given $n$-r.e. set $%
W=\Phi (A)$ from one $G_{i}$ that we are building over the full
construction. The most delicate part of the verification of the first
construction is the correctness of these functionals (Lemma \ref{Delta}). We
argue that the cause of an incorrect computation, say of $\Delta (u)$, must
be that some number $z$ entered $A$ for the first time and allowed $W(u)$ to
change. Another delicate argument shows that if $W$ also changed for the
first time, we could correct the functional $\Delta $ (or see that we are
not on the true path). If the change in $W$ was not that $u$ entered for the
first time, we argue that we can shuffle $A$ between two past values
(giving, via $\Phi $, two different values for $W(u)$) by repeatedly taking $%
z$ out and putting it back in as necessary so as to eventually show that $%
W\neq \Phi (A)$.The point here is that $z$ has entered $A$ for the first
time while the change in $W$ is not a first change. Thus as $W_{l}$ can make
no more than $n$ changes overall, it can make no more than $n-2$ additional
changes. On the other hand, as $z$ has entered $A$ for the first time, we
can make $n-1$ more changes in $A$ and so eventually guarantee that $W\neq
\Phi (A)$.

In the second construction (\S \ref{Plan R}), the correctness of the
functionals $\Delta $ becomes immediate as we simply change $G$ when
necessary. The crucial problem then becomes guaranteeing the correctness of
computations from $G$ diagonalizing against $D$ (\S \ref{ver II}). Here we
take advantage of the fact that $D$ can change one more time than any other
set by using a procedure like one used in Yang and Yu [2006] to remove a
number (that entered for the first time) from $D$. In our case it allows us
to either cure some problem we are facing or start a shuffling procedure for 
$A$ diagonalizing against the offending $W$.

\section{Basic Notions and Conventions\label{basic}}

Given a set $A$, let $A\upharpoonright u$ be the initial segment of $A$ of
length $u$.

We use upper case Greek letters to denote Turing functionals. For any Turing
functional $\Delta $, the \emph{use} of a convergent $\Delta (A;x)$ is
defined as the least number $u$ such that $\Delta (A\upharpoonright
u;x)\downarrow $. We use lower case Greek letters corresponding to the
Turing functional to denote the use, e.g. $\delta (A;x)$ denotes the use of $%
\Delta $ at $x$. More importantly, we injure the computation by adding $%
\delta (A;x)-1$ into $A$, but not by adding $\delta (A;x)$ into $A$. If it
doesn't cause confusion, we may omit $A$ and write $\delta (x)$.

We will have families $\Psi $, $\Pi $, $\Theta $ and $\Phi $ which specify
standard enumerations of all the Turing functionals. We follow the usual
conventions for such standard enumerations such as the approximations to
these functionals for any (approximation to an) oracle set at stage $s$ asks
questions about (makes use of) only numbers less than $s$ and converges only
at inputs less than $s$. We also assume, without loss of generality, that
for the standard enumerations with two oracles such as $\Theta (G\oplus P;x)$
the uses on both are always the same and we denote it by $\theta (x)$.

We will also construct two families of Turing functionals $\Delta (G)$ and $%
\Gamma (W\oplus P)$. For the ones with two oracles, we \emph{do not} require
that the uses of $W$ and $P$ are the same. Hence we can write $\gamma (W;x)$
and $\gamma (P;x)$ to denote the $W$ and $P$ parts of the use, respectively.
Although for simplicity we generally work as if we are specifically defining
these oracles at each individual $x$ with the associated uses, we really are
assuming that the uses are monotonic in $x$ and make all changes to keep
them that way, usually without explicit mention. As $Q$ is r.e., when we are
computing it from $W\oplus P$ by $\Gamma $, except for this monotonicity
condition, we only need to produce computations (axioms) that at $x$ give
output $0$ when $x\notin Q$. These may be injured and new ones put into $%
\Gamma $ (perhaps with larger use). In the case that $x\notin Q$ and we are
expecting $\Gamma $ to compute $Q$, we must eventually settle on a
convergent computation (axiom) applying to $W_{i}\oplus P$. If $x\in Q$,
when $x$ enters $Q$ it suffices to kill any current computation of $0$ from $%
W_{i}\oplus P$. We do this by putting a number less than the $P$-use into $P$%
. We can then simply keep the value of $\Gamma $ at $1$ without changing the
use (remembering that $P$ is r.e.).

In our two constructions, we specify priority trees which grow downward. At
each stage $s$ of the construction, we build a path of length $s$ \{at most $%
s$\} of \emph{accessible} nodes along the priority tree. Our convention is
that, the nodes to the left of, or above, a node $\alpha $ have higher
priority. We always preserve the information used at previous stages by the
nodes that are to the left of the accessible ones by initializing the nodes
that are to the right of the accessible ones, i.e., remove all information
from previous stages such as witness numbers, defined functionals and
imposed restraints.

Nodes can impose two types of restraint: a permanent one or an alternating
one. Permanent restraint means that no node of lower priority can act so as
to injure the restraint by changing a set where restrained. By convention
permanent restraint imposed at stage $s$ restrains the initial segments of
length $s$ of $L$ and all the $G_{i}$ \{$A$, $D$ and $G$\}. Any permanent
restraint on $P$ must be mentioned specifically. [We never need to restrain $%
Q$.] Alternating restraints are caused by the announcements of $A$-stages or 
$P$-stages which we describe later in the construction. Basically, during $A$%
-stages, we remove the alternating restraint for $L$ and the $G_{i}$ \{$A$
and $D$\} allowing numbers to enter (or leave) these sets and we impose an
alternating restraint on $P$ and $Q$ \{and $G$\} so that no numbers can
enter $P$ or $Q$ \{or $G$\} at this stage. During $P$-stages, we do the
opposite (except that no numbers ever leave the r.e. sets $P$ or $Q$ \{or $G$%
\}).

\section{Requirements I\label{Req}}

\label{t1s1}

We now begin the proof of our main technical result. \setcounter{section}{1} %
\setcounter{theorem}{3}

\begin{theorem}
Given a recursive partial order $(\omega ,\leq _{\ast })$ and an $n\geq 1$,
there exist uniformly $n$-r.e.\ sets $G_{i}$ for each $i\in \omega $, an $n$%
-r.e.\ set $L$ and r.e.\ sets $P$ and $Q$ such that:

\begin{enumerate}
\item Each $\mathbf{g}_{i}$ is a maximal $n$-r.e. degree below $\mathbf{a}$
such that $\mathbf{q}\nleq \mathbf{g}_{i}\vee \mathbf{p}$ where $%
A=\bigoplus_{i}G_{i}$.

\item $\mathbf{g}_{i}\leq \mathbf{g}_{j}\vee \mathbf{l}$ if and only if $%
i\leq _{\ast }j$.
\end{enumerate}
\end{theorem}

\setcounter{section}{3} \setcounter{theorem}{0} First, for the negative
order facts, we have requirements for each pair $i\nleq _{\ast }j$ and each $%
e$: 
\begin{equation*}
\Psi _{e,i,j}:\ \Psi _{e}(L\oplus G_{j})\neq G_{i}.
\end{equation*}

Similarly for each triple $(i,j,e)$ with $i\neq j$, we also want: 
\begin{equation*}
\Pi _{e,i,j}:\ \Pi _{e}(G_{j})\neq G_{i},
\end{equation*}

i.e., the $G_{i}$'s are pairwise incomparable.

Then for each pair $(i,e)$ we need: 
\begin{equation*}
\Theta _{e,i}:\ \Theta _{e}(G_{i}\oplus P)\neq Q.
\end{equation*}

We also need the main requirements that each $\mathbf{g}_{i}$ is a maximal $%
n $-r.e. degree $\mathbf{g\leq _{T}a}$ such that $\mathbf{q}\nleq \mathbf{g}%
\vee \mathbf{p}$. We let $W_{i}$ be an effective list of all the $n$-r.e.
sets. 
\begin{equation*}
\Phi _{e,i}:\ \Phi _{e}(A)=W_{i}\rightarrow \lbrack \exists \Gamma (\Gamma
(W_{i}\oplus P)=Q)\vee (\exists k(W_{i}\leq _{T}G_{k}))].
\end{equation*}

\begin{itemize}
\item Note that these $\Phi $ requirements by themselves do not ensure that
each $\mathbf{g}_{i}$ is maximal. That is why we need the $\Pi $
requirements to make all the $G_{i}$'s pairwise incomparable. The $\Phi $
requirements then do guarantee that the $\mathbf{g}_{i}$ are maximal.
\end{itemize}

If it does not cause confusion, we may omit the subscripts of the
requirements and sets in our argument to simplify the notation.

Finally, we have to deal with the positive order facts, i.e., $G_{i}\leq
_{T}L\oplus G_{j}$ for $i<_{\ast }j$. We will guarantee that, for $x>i,j$, $%
x\in G_{i}\Leftrightarrow x\in L$ or $x\in G_{j}$. Putting numbers into a $%
G_{i}$ is initiated only by a $\Psi $ or $\Pi $ requirement. For $\Pi $
action, we simply put a witness $x$ that is going into $G_{i}$ (for
diagonalization) into $L$ as well. When action is initiated for
diagonalization by $\Psi $ at stage $s$, we put $x$ into $G_{i}$ and also
into each $G_{l}$ with $l>_{\ast }i$ for each $l<x$. As, in this case, $%
i\nleq _{\ast }j$, this action does not add elements to $G_{j}$ and so it
does not injure the $\Psi $ computation initiating the action. We say that
each witness $x$ (for a $\Psi $ or $\Pi $ requirement) has an \emph{%
associated block of sets} (the $G_{l}$ such that $l<x$ and $i\leq l$ or $%
G_{i}$ and $L$, respectively). During the construction $x$ moves into or out
of all the sets in its block simultaneously.

\section{Priority Tree I\label{tree}}

We put all the $\Psi ,\Pi ,\Theta $ and $\Phi $ requirements into one
priority list. Our \emph{priority tree} consists of \emph{nodes} and \emph{%
branches}. Each node is associated with a requirement in the list and each
branch leaving a node is assigned an outcome. We label each node with its
associated requirement and each branch with the assigned outcome. When we
list outcomes of a node we do so in a left to right order that specifies the
left to right order on the priority tree of the branches leaving that node

A $\Psi $ or $\Pi $ node has two outcomes: $d$ and $w$, which stand for
\textquotedblleft diagonalization\textquotedblright\ and \textquotedblleft
wait\textquotedblright\ respectively.

A $\Phi $ node has outcomes $s_{n-1}$, $s_{n-2}$, ..., $s_{1}$, $i$ and $w$.
Outcome $s_{i}$ stands for \textquotedblleft shuffle\textquotedblright\ for
the $i$-th time. We will explain what this means in detail in the
construction. Roughly, it means that we expect to shuffle between two
versions of $A$ (by removing numbers from $A$ and then possibly putting them
back in) as we cycle back to this node. The expected result of this
shuffling is to guarantee that $\Phi (A)\neq W$ by a diagonalization.
Outcome $i$ stands for \textquotedblleft infinite\textquotedblright\
agreement between $\Phi (A)$ and $W$ and outcome $w$ stands for
\textquotedblleft wait\textquotedblright .

A $\Theta $ node $\beta $ has outcomes $d$, $g_{\alpha _{1}}$, $g_{\alpha
_{2}}$,...,$g_{\alpha _{k}}$ and $w$. As usual, $d$ and $w$ stand for
\textquotedblleft diagonalization\textquotedblright\ and \textquotedblleft
wait\textquotedblright\ respectively. Each $\alpha _{i}$ is a $\Phi $ node
above $\beta $ which has outcome $i$ along $\beta $. If $\gamma $ is a node
below $\beta $ extending the $g_{\alpha _{i}}$ branch from $\beta $, then we
say that $\gamma $ \emph{sees} an $\alpha _{i}-\beta $ pair. [The intuition
here is that $\gamma $ believes that $\alpha _{i}$ and its associated
requirement is satisfied by $\beta $.]

A $\Phi $ node $\alpha $ is \emph{active} at $\gamma \supset \alpha $ if $%
\alpha $ has outcome $i$ along $\gamma $ and $\gamma $ does not see an $%
\alpha ^{\prime }-\beta ^{\prime }$ pair such that $\alpha ^{\prime
}\subseteq \alpha \subset \beta ^{\prime }\subset \gamma $. For there to be
a $g_{\alpha _{i}}$ outcome of a $\Theta $ node $\beta $, we also require
that $\alpha _{i}$ be active at $\beta $. We order these $g_{\alpha _{i}}$'s
from left to right in descending order going down the tree to $\beta $,
i.e., $\alpha _{1}\subset \alpha _{2}\subset ...\subset \alpha _{k}$. [This
choice of left to right order comes into play at the very end of the proof
of Lemma \ref{Phi i} and is discussed in \S \ref{s13}.]

The priority tree is defined recursively as follows: suppose $\tau $ is an
immediate extension of $\sigma $, we associate $\tau $ with the highest
priority requirement among all requirements which either have not appeared
above $\tau $ or are $\Phi $ requirements that, above $\tau $, have appeared
only at nodes $\delta $ with outcome $i$ such that $\tau $ sees an $\alpha
-\beta $ pair with $\alpha \subset \delta \subset \beta $. [So $\delta $
looks inactive but not really satisfied, i.e. if satisfied at some earlier
point it has since been \textquotedblleft captured\textquotedblright\ by
some other pair.] Then we add the corresponding number of branches
(outcomes) below $\tau $. It is easy to see that this tree is recursive.

\section{Construction I\label{con}}

At stage $s$ of the construction, we build a path of length $s$ of the \emph{%
accessible nodes }along the priority tree. It is possible that at some
accessible node we will \emph{announce }that $s$ is an $A$-stage or a $P$%
-stage. All later nodes accessible at $s$ must respect this announcement by
acting according the the rules governing $A$-stages or $P$-stages: no
changes in $A$ can occur once a $P$-stage has been announced and none in $P$
or $Q$ once an $A$-stage has been announced. In the construction, we will
make sure that the first accessible $\Theta $ node with a type $g$ outcome
(if any) makes the announcement for the stage $s$. An over-riding rule is
that permanent restraint imposed by a node (not since initialized) is not
violated by action at any node of lower priority (i.e. below it or to its
right). If any instruction below leads to any such situation, we do not
carry it out and instead go to outcome $w$ [and do nothing].

In this section, we first describe the construction at stage $s$ for each
node when there has been as yet no announcement for the stage and then
specify the modifications for when there has already been one.

\subsection{no announcement, $\Psi$ or $\Pi$ node}

\label{t1s3c1}

The actions at $\Psi $ and $\Pi $ nodes are quite standard: If it is the
first time we come to this node (after it was last initialized), then we
pick a witness number $x$ which is \emph{fresh}, i.e., larger than any
number we have seen by this point in the construction. In general, at a $%
\Psi $ ($\Psi (L\oplus G_{j})\neq G_{i}$) or a $\Pi $ ($\Pi (G_{j})\neq
G_{i} $) node with a witness $x$ already assigned (and not yet canceled by
initialization), we check whether the computation at $x$ converges to $0$.
If it diverges or converges to a nonzero number, then we do nothing and go
to the $w$ outcome. If it converges to $0$ and $x\notin G_{i}$, then we do a
diagonalization: put $x$ into $G_{i}$ and into all the other sets in its
block as described in Section \ref{t1s1}, impose permanent restraint [to
preserve the use of the computation] and go to the $d$ outcome. If $x$ is
already in $G_{i}$, then we (again) go to outcome $d$ [and keep the
restraint already imposed].

\subsection{no announcement, $\Phi $ node\label{shuf}}

\label{t1s3c2}

At a $\Phi $ node $\alpha $ ($\Phi (A)=W$) if we have not yet had a type $s$
outcome (since $\alpha $ was last initialized) let $t$ be the last stage at
which $\alpha $ was accessible (since last initialized). If there is a such
stage and a $u<l_{\alpha }(t),l_{\alpha }(s)$ such that $\Phi (u)$ (and so $%
W(u)$) differ at $t$ and $s$ with the difference not being that $u$ has
entered $W$ for the first time and the only change in $A\upharpoonright \phi
(u)$ at $t$ is that some $z$ has entered its block of sets for the first
time because of the action of a node extending $\alpha $ then we \emph{%
initiate a shuffle on }$z$\emph{\ }by removing $z$ from its block of sets,
impose permanent restraint and go to outcome $s_{1}$. We call this shuffle
strategy \emph{Plan S} with \emph{shuffle points }$sp1(=t)<sp2(=s)$. [Note
that these shuffle points have the property that $A_{sp1}$($%
=A_{sp1}\upharpoonright sp1$) and $A_{sp2}$($=A_{sp2}\upharpoonright sp2$)
differ below $sp1$ only in that $z$ is in its block of sets in $A_{sp2}$ and
out of them in $A_{sp1}$. More crucially, they produce different values for $%
\Phi $ at some $u$, i.e. $\Phi (A_{sp1};u)\neq \Phi (A_{sp2};u)$.] If we had
an outcome of type $s$ at the last stage $t$ at which $\alpha $ was
accessible, we check whether $W(u)$ is different at $s$ than at $t$. If so,
restore the initial segment of $A$ to the version of $A$ which is different
from the current one (by putting $z$ into or taking it out of its block of
sets), impose permanent and let the outcome be $s_{i+1}$. If not, we stay at
the $s_{i}$ outcome. [This maintains any previously imposed permanent
restraint.]

If we haven't initiated shuffling, let $l_{\alpha }(s)$ be the length of
agreement between the current versions of $\Phi (A)$ and $W$. Note that
whenever we initialize this node $\alpha $, we also initialize the values of
this function to be $0$. If this is the first time that $l_{\alpha }(s)>0$
after it has last been initialized, or $l_{\alpha }(s)>l_{\alpha }(t)$ where 
$t$ is the last stage when $\alpha $ had an $i$ outcome, then we go to the $%
i $ outcome; otherwise we go to the $w$ outcome.

If we go to the $i$ outcome, we continue to define a functional $\Gamma $
[aiming to make $\Gamma (W\oplus P)=Q$]. At this point, we enumerate a new
axiom making $\Gamma (W\upharpoonright l_{\alpha }(s)\oplus P\upharpoonright
v;w)=Q(w)$, where $v$ is a fresh number and $w$ is one more than the largest
number where we have previously defined $\Gamma $ (since it was last
initialized). If $P$ has changed on its $\Gamma $-use at some $x<w$ and the
change was caused by the action of a node $\beta \symbol{94}g_{\alpha }$
[necessarily extending $\alpha \symbol{94}i$] with witness $x$ as in \S \ref%
{t1s3c3ss3}, then we redefine $\Gamma (x)$ to be the current value of $Q(x)$
with $W$-use $l_{\alpha }(s)$ and fresh $P$-use. [As $P$ is r.e. this change
permanently invalidates the previous axiom for $\Gamma (x)$.] Similarly, if $%
W$ has changed on its use $u_{1}$ (where its old $P$-use is $v_{1}$) so as
to make $\Gamma (x)$ divergent but $x$ has not entered $Q$, we see if $x$ is
currently the witness for some $\Theta $ node $\beta $ below $\alpha $ (for $%
G$). If so, we look at the last stage $t$ at which $\beta $ was accessible
and see if its outcome was $g_{\alpha }$. If $G$ has not changed on $\theta
(x)$ as defined at the point of stage $t$ at which $\beta $ was reached and
the change in $W$ includes one at some $u$ making it different from the
common value of $\Delta (u)$ and $W(u)$ at $t$, then we redefine $\Gamma (x)$
with $W$-use $l_{\alpha }(s)=u_{2}$ and fresh $P$-use. In all other cases of
a $W$ or $P$ change on $\gamma (W;x)$ or $\gamma (P;x)$, respectively, that
makes $\Gamma (x)$ divergent we redefine $\Gamma (x)$ with the same uses as
it last had but for the new values of $W$ and $P$ (subject, of course, to
our monotonicity requirements on the use).

\subsection{no announcement, $\Theta $ node}

\label{t1s3c3}

At a $\Theta $ node $\beta $ accessible for first time after it has been
last initialized, we pick a fresh witness $x$ for diagonalizing $\Theta
(G_{i}\oplus P)\neq Q$. In general, if we have a witness $x$ already
assigned (and not yet canceled by initialization), we check whether the
computation converges at the witness $x$. [As usual when there are higher
priority requirements that are expected to put infinitely many numbers into
a set, we restrict our attention to computations that are consistent with
our beliefs as prescribed by our actions in \S \ref{t1s3c3ss3}. Here this
means the following:] We also require that the computation be \emph{%
believable}, i.e. for every requirement $\hat{\Theta}$ assigned to a node $%
\alpha $ with witness $\hat{x}$ and $\alpha \symbol{94}g_{\hat{\alpha}%
}\subseteq \beta $ for some $\hat{\alpha}$, $\theta (x)<\gamma _{\hat{\alpha}%
}(P;\hat{x})$ and if $\gamma _{\hat{\alpha}}(P;\hat{x})$ has been previously
increased by a $\hat{W}$ change (as described at the end of \S \ref{t1s3c2})
from say $u_{1}$ to $u_{2}$ and $v_{1}-1$ is not yet in $P$ then $\theta
(x)<v_{1}$ as well. If $\Theta (x)$ does not converge with a believable
computation or so converges to a nonzero number, then we go to the $w$
outcome and do nothing.

[If the believable computation $\Theta (G_{i}\oplus P;x)$ converges to $0$
with $P$-use $\theta (x)$, then we would like to diagonalize, i.e., put $x$
into $Q$ and preserve the $P$ and $G_{i}$ use of the computation. However,
we must worry about whether doing so injures some already defined $\Gamma $
computation at a node above $\beta $. For example, if there is a such a $%
\Gamma (W\oplus P)=Q$ which computes $Q(x)=0$ with $\gamma (P;x)\leq \theta
(x)$, then our desired diagonalization would falsify this computation of $Q$
while correcting the $\Gamma $ computation (by putting its use into $P$ and
redefining the functional) would injure our $\Theta $ computation for
diagonalization. Our plans must be more subtle.] If $\Theta (x)$ converges
with a believable computation we proceed as follows:

Let $\alpha _{1}\subset \alpha _{2}\subset ...\subset \alpha _{k}$ be all
the active nodes above $\beta $ with each $\alpha _{j}$ defining its
functional $\Gamma _{j}(W_{l_{j}}\oplus P)$. Let $\gamma _{j}(P;x)$ be the $%
P $-use of $\Gamma _{j}$ at $x$, if it has already been defined. [See \S \ref%
{s13} for some comments on the left-right ordering indicted here for these
nodes and the associated outcomes $g_{\alpha _{j}}$ below.]

\subsubsection{Plan D: diagonalization}

\label{t1s3c3ss1}

If $\theta (x)<\gamma _{j}(P;x)$ for all $j$ for which $\gamma _{j}(P;x)$ is
defined, we do a modified diagonalization: We enumerate $x$ into $Q$ and
also enumerate $\gamma _{\alpha _{j}}(P;x)-1$ into $P$ for each $j$. [This
allows us to correct the $\Gamma _{j}(x)$ when $\alpha _{j}\symbol{94}i$ is
next accessible.] We now impose the usual permanent restraint but also one
on $P\upharpoonright \theta (x)$ [to preserve the $\Theta $ computation] and
go to outcome $d$. Until $\beta $ is initialized, it has outcome $d$ at
every later stage at which it is accessible.

\subsection{Stage Announcements}

If we cannot follow Plan D, i.e., $\theta (x)\geq \gamma _{j}(P;x)$ for some 
$j$, then we take the largest $j$ such that $\theta (x)\geq \gamma _{j}(P;x)$
[and are likely to go to outcome $g_{\alpha _{j}}$ where we build a
functional $\Delta $ computing $W_{l_{j}}$ from $G_{i}$]. [The choice of $j$
is relevant at the very end of the proof of Lemma \ref{Delta} but our choice
of the largest $j$ (rather than say the smallest) doesn't make any
difference in this construction. It does, however, matter in the at the end
of the proof of Lemma \ref{Plan R lem} for our second theorem.]

\subsubsection{Plan A: $A$-stage announcement}

\label{t1s3c3ss2}

If this is the first time (since the last initialization) that we would go
to the $g_{\alpha _{j}}$ outcome or the last time we went there we announced
a $P$-stage then we go to outcome $g_{\alpha _{j}}$ and announce an $A$%
-stage [and so allow elements to be enumerated into or taken out of $A$].

Otherwise, let $t$ be the last stage when $\beta \symbol{94}g_{\alpha _{j}}$
was accessible. By our construction and case assumption, $t$ must have been
announced as an $A$-stage at $\beta \symbol{94}g_{\alpha _{j}}$. (If some
node to the right or left of $\beta \symbol{94}g_{\alpha _{j}}$ made an
announcement at stage $t$ then $\beta $ would not have been accessible at $t$%
. If some node $\alpha $ above $\beta $ made an announcement at $t$ then one
would also have to be made above $\beta $ at $s$ contrary to our case
assumption that no announcement has been made at this stage before we
reached $\beta $.)

%\subsubsection{Plan S: shuffle}
%
%\label{t1s3c3ss4}
%
%If we see that $W_{l_{j}}(u)$ is no longer the value previously computed by $%
%\Delta $ at $t$ for some $u$ and the only change in $A\upharpoonright t$
%since $t$ is that some $z$, a witness for a node $\alpha \supset \beta $,
%has entered its block of sets for the first time while the change in $%
%W_{l_{j}}(u)$ is one other than $u$ entering for the first time, we \emph{%
%initiate the shuffling strategy for }$\alpha _{j}$ as follows: Let $%
%A_{sp1}=A_{t}\upharpoonright t$ and $A_{sp2}=A_{s}\upharpoonright s$. So we
%have $\Phi _{j}(A_{sp1};u)\downarrow \neq \Phi _{j}(A_{sp2};u)\downarrow $.
%We impose permanent restraint [to preserve $A_{sp2}$] and move to outcome $w$%
%. [The rest of the shuffling plan will then be taken over by $\alpha _{j}$
%(as described above for $\Phi $ nodes) once it is accessible again.] If not,
%we move to the next plan.

\subsubsection{Plan P: $P$-stage announcement}

\label{t1s3c3ss3}

We now go to the $g_{\alpha _{j}}$ outcome and extend $\Delta $ by adding
axioms computing $W_{l_{j}}(u)$ from $G_{i}$ with fresh use for any $%
u<l_{\alpha _{j}}(s)$ for which $\Delta $ has not previously been defined.
In addition, we put $\gamma _{j}(P;x)-1$ into $P$ to injure the current $%
\Theta $ computation (since $\gamma _{j}(P;x)\leq \theta (x)$). [This kills
the current computation of $\Gamma (x)$ and as $P$ is r.e. it can never
apply to $W\oplus P$ again.] Moreover, if $\gamma _{j}(P;x)$ has been
previously increased by a $\hat{W}$ change (as described at the end of \S %
\ref{t1s3c2} from say $u_{1}$ to $u_{2}$ and $v_{1}-1$ is not yet in $P$
then we also put $v_{1}-1$ into $P$. [This kills the old computation of $%
\Gamma _{j}(x)$ as well as and guarantees that it too will never again apply
to $W\oplus P$.] [We will redefine $\Gamma _{j}$ with axioms using the new
version of $P$ with a fresh $P$-use and $W_{l_{j}}$ use $l_{\alpha _{j}}(v)$
when we next get to $\alpha _{j}\symbol{94}i$ at $v$. The result of this
action is that we increase the $\Gamma _{j}$ use from $P$ and $W_{l_{j}}$
and so the next time when this $\beta $ is accessible with the $g_{\alpha
_{j}}$ outcome, the use $\theta (x)$ must be larger than that of this stage.
If this happens infinitely often $\Gamma _{j}(x)$ diverges but we expect to
satisfy the associated $\Phi $ requirement by building $\Delta
(G_{i})=W_{l_{j}}$ at $\beta \symbol{94}g_{\alpha _{j}}$. We then also
satisfy the $\Theta $ requirement associated with $\beta $ as $\Theta (x)$
diverges as well.] We now announce that the current stage is a $P$-stage.

If there has been a change in $G_{i}$ that leaves $\Delta (u)$ undefined
where it had previously been defined, we put in a new axiom computing the
current value of $W_{l_{j}}(u)$ with the old use.

[We shall argue for $\beta $ on the true path with true outcome $g_{\alpha
_{j}}$ that we build $\Delta $ consistently and correctly compute $W_{l_{j}}$
at each stage (Lemma \ref{Delta}). Typically, it turns out that, along the
true path, if $W_{l_{j}}$ has changed where previously computed, then $G_{i}$
must have changed at the corresponding part used in the computation.]

\subsection{modifications with a stage announcement}

\label{t1s3c4}

When there is has already been a stage announcement before we reach $\beta $%
, the node $\beta $ has to obey the appropriate rules. For a $\Psi $, $\Pi $
or $\Theta $ node, we see what we would have done if there had been no stage
announcement as yet. If that action is compatible with the current stage
announcement (no announcement of an $A$-stage or change in $A$ if a $P$%
-stage; no change in $P$ or $Q$ and no announcement of a $P$-stage if an $A$%
-stage), we proceed as if there had been no announcement. If not, we do
nothing and go to outcome $w$.

For a $\Phi $ node, the modification is slightly trickier. [Later we will
need the fact that each node along the true path passes down alternating $A$
and $P$ restraints in the construction.] Here in order to go to the $i$
outcome, we need to wait (with outcome $w$) for a stage when the stage
announcement is different from the last stage $t$ when we had an $i$
outcome, and also the length of agreement is longer than its last value. [In
this way, the $\Phi $ node passes down alternating $A$ and $P$ restraints
along the $i$ outcome.] When we have already initiated shuffling, we act as
before at $A$-stages and at $P$-stages we go to outcome $w$. [This maintains
the permanent restraint imposed when we initiated shuffling or last shuffled
as the nodes that imposed it are now to our left.]

\section{Verification I\label{ver}}

\label{t1s5}

\subsection{True path and true outcome}

\label{t1s4ss1}

First of all, as in usual priority tree arguments, there is a leftmost path
accessible infinitely often. (Each node has only finitely many outcomes.)
This is the \emph{true path} and the outcomes along it the \emph{true
outcomes}.

\begin{lemma}
\label{change A}Numbers enter or leave $A$ or $L$ only when permanent
restraint is imposed by a $\Pi $, $\Psi $ or $\Phi $ node. When such nodes $%
\beta $ impose permanent restraint, we move to the left of any previous
outcome that has been accessible since $\beta $ was last initialized.
\end{lemma}

\begin{proof}
By inspection of the construction.
\end{proof}

\begin{lemma}
\label{only one}At most one node acts to change $A$ at any stage $s$.
\end{lemma}

\begin{proof}
If we first act at $\alpha $ to change $A$ at $s$ then we move to an outcome
to the left of all previously accessible ones (since $\alpha $ was last
initialized) by Lemma \ref{change A}. So all later nodes accessible at $s$
that can change $A$ are accessible for the first time since last initialized
and so at most appoint fresh witnesses or (for $\Phi $ nodes) begin their
construction of $\Gamma $ anew. None of these witnesses can go in at $s$ as
no convergences can be seen at numbers larger than $s$. No shuffling can
been initiated for any of the $\Phi $ nodes by construction.
\end{proof}

\begin{lemma}
\label{after init}If a node $\alpha $ is initialized at stage $s$ then it
never later acts to change any set below $s$.
\end{lemma}

\begin{proof}
If $\alpha $ is a $\Pi $, $\Psi $ or $\Theta $ node it only acts to put
numbers at least as large as its witness $x$ into $A$ or $P$ and any witness
appointed after $s$ is larger than $s$. For $\Pi $ and $\Psi $ nodes this is
immediate. For $\Theta $, its action puts numbers of the form $\gamma
(P;x)-1 $ into $P$ and by construction $\gamma (P;x)>x$. For $\Phi $ nodes,
the only action changing sets is shuffling. This shuffling only involves
numbers appointed below $\alpha $ at stages when $\alpha $ was accessible
since it was last initialized.
\end{proof}

Recursively along the true path, we now determine the actions of the nodes
on it after no node to their left is ever accessible and prove that all the
requirements are satisfied along it. For any node $\beta $ on the true path
we let $s(\beta )$ be the first stage at which $\beta $ is accessible but
after which no node to its left is ever accessible again.

\begin{lemma}
\label{rest A}Any permanent restraint imposed by a node $\beta $ at any $%
s\geq s(\beta )$ is never injured by any other node.
\end{lemma}

\begin{proof}
The only actions that can injure such restraint after $s(\beta )$ are ones
by nodes above it on the true path. None can change $A$ or $L$ by Lemma \ref%
{change A}. As for $P$, the only permanent restraint imposed on $P$ is by $%
\Theta $ nodes when we go to outcome $d$ and restrain $P\upharpoonright
\theta (x)$. Now nodes $\hat{\beta}$ above $\beta $ of type $\Theta $ with
outcomes $g_{\hat{\alpha}}$ may put numbers into $P$ but they only put in
ones of the form $\gamma _{\hat{\alpha}}(P;\hat{x})$ and our believability
requirement on the computation of $\Theta (x)$ guarantees that all of the
current values of these $\gamma _{\hat{\alpha}}(P;\hat{x})$ are larger than $%
\theta (x)$. The only way one of them could decrease is if it had previously
been increased from $u_{1}$ to $u_{2}$ by a change in $W$ as described at
the end of \S \ref{t1s3c2} and then $W$ changes back to the old value before
the old computation is killed by $v_{1}-1$ going into $P$. However, our
believability condition also requires that $\theta (x)$ is less than these $%
v_{1}$ as well. Any later change increases the use above the previous values
Thus no changes every occur in $P$ below $\theta (x)$.
\end{proof}

\begin{lemma}
\label{wit large}The final witness for any node $\beta $ chosen at $s(\beta
) $ is larger than any permanent restraint of higher priority than $\beta $.
\end{lemma}

\begin{proof}
By construction the witness is chosen fresh and so larger than anything
previously seen. The only nodes of higher priority that can impose permanent
restraint later are ones above $\beta $. None of type $\Pi $, $\Psi $ or $%
\Phi $ can do so by Lemma \ref{change A}. One of type $\Theta $ also imposes
permanent restraint only when it moves left to outcome $d$ contradicting our
definition of $s(\beta )$.
\end{proof}

Before we show that the requirements are satisfied we analyze the
alternating restraint.

\subsection{Alternating $A$ and $P$-stages}

\label{t1s4ss2}

\begin{lemma}
\label{alt}Every node along the true path above the first $\Theta $ node $%
\beta $ with type $g$ outcome on the true path never sees or makes a stage
announcement (imposes alternating restraint) when accessible. For the other
nodes $\alpha $ on the true path, their true outcomes, $o$, are accessible
at infinitely many $A$ and $P$-stages. Indeed, after $s(\alpha \symbol{94}o)$%
, the stages at which $\alpha \symbol{94}o$ is accessible alternate between $%
A$ and $P$ ones (the node passes down alternating $A$ and $P$ restraints
along its true outcome).
\end{lemma}

\begin{proof}
For any node above $\beta $ the claim is immediate from the rules of the
construction. For a $\Pi $ or $\Psi $ node below $\beta $, it is immediate
from the construction that after $s(\beta )$ either we always have outcome $%
w $ or, whenever we are at $\beta $ after the first time we have outcome $d$
we also have outcome $d$. So for these nodes the Lemma is obvious. For a $%
\Phi $ node below $\beta $, either the true outcome is shuffling ($s$), or
waiting ($w$), or infinitary ($i$). In the two former cases, the outcome is
again eventually constant: Once we move to a type $s$ outcome, the
construction guarantees that we can move only to the left to another type $s$
outcome. Thus the outcome is eventually constant at some type $s$ outcome.
As for outcome $w$, any rightmost outcome that is the true outcome of a node 
$\beta $ on the true path is the outcome at almost every stage at which $%
\beta $ is accessible. In the third case, our construction in \S \ref{t1s3c4}
ensures that it passes down alternating $A$ and $P$ restraint along the
outcome $i$ as required.

The $\Theta $ node $\beta $ which first makes the announcements along the
true path must have true outcome some $g_{\alpha }$. Then according to our
construction, if it announced a $P$-stage the last time it was accessible,
we follow Plan $A$ and make an $A$-stage announcement. If it last announced
an $A$-stage, then we follow Plan P and announce a $P$-stage.

% since acting for the
%only other possible plan (shuffle) would make us move to the $s_{1}$
%outcome for $\alpha $ when it is next accessible but $\alpha \symbol{94}%
%s_{1} $ is to the left of $\alpha \symbol{94}i\subseteq \beta $ for a
%contradiction.

Finally for any other $\Theta $ node $\beta ^{\prime }$ on the true path,
the claim is also immediate for true outcome $d$ or $w$ as above. In the
case of a $g_{\alpha }$ outcome, the construction automatically guarantees
that it always waits for an alternation in the type of restraint to move
again to the true $g_{\alpha }$ outcome.
\end{proof}

We next analyze the functionals $\Delta $ and $\Gamma $ that we construct.

\subsection{The functionals are well-defined and correct\label{t1s4ss3}}

\begin{lemma}
\label{Gamma}The functionals $\Gamma $ built at nodes $\alpha \symbol{94}i$
on the true path starting at $s(\alpha \symbol{94}i)$ are well-defined,
i.e., we do not add contradictory axioms in the construction and when
defined give the correct current value for $Q$. They are defined on
arbitrarily large initial segments of $\omega $ and so, if convergent at
every $x$, they correctly compute the desired sets.
\end{lemma}

\begin{proof}
It is clear by construction that we define $\Gamma (x)$ at least once for
each $x$: As $\alpha \symbol{94}i$ is on the true path, $l_{\alpha }(s)$ is
going to infinity on the stages at which $\alpha \symbol{94}i$ is
accessible. On each of these stages we define $\Gamma $ on a new number.
Once defined $\Gamma (x)$ is then defined at every stage at which $\alpha 
\symbol{94}i$ is accessible by construction. As for consistency and
correctness, note first that it is immediate from the construction that at
any stage at most one axiom in $\Gamma $ applies to the current value of $%
W\oplus P$. Now, the only way any problem could arise is if $Q(x)$ has
changed for some $x$ but $W$ and $P$ have not changed on the corresponding
use or they change and then $W$ reverts back to a previous value that
applies to some older computation (with value $0$). However, in our
construction, $Q(x)$ can change only when we implement Plan D to put $x$
into $Q$ at a some stage $v$ for a $\Theta $ node $\hat{\beta}$. Any number $%
x$ put into $Q$ by such nodes to the right of $\alpha \symbol{94}i$ must be
both appointed fresh and then put in while we are to the right of $\alpha 
\symbol{94}i$ without $\alpha \symbol{94}i$ becoming accessible in between.
Thus when $\alpha \symbol{94}i$ is again accessible $\Gamma $ has not been
defined at $x$ and so when we define $\Gamma (x)$ we set it equal to $1$
with the first axiom and all later ones as well. Any $x$ put in by nodes to
the left of $\alpha \symbol{94}i$ are in by $s(\alpha \symbol{94}i)$ and so
we define $\Gamma $ correctly on them as well.

Thus we can assume that $\hat{\beta}$ is below $\alpha \symbol{94}i$. If $%
\alpha $ is active at $\hat{\beta}$, then when $x$ enters $Q$ at $v$ we put $%
\gamma (P;x)-1$ into $P$ by construction and so kill the current computation
and put in a new one giving the correct answer when we next reach $\alpha 
\symbol{94}i$. If $\alpha $ is not active at $\hat{\beta}$, there must be a
first $\beta ^{\prime }\subset \hat{\beta}$ with $\beta ^{\prime }\symbol{94}%
\hat{\alpha}\subset \hat{\beta}$ for some $\hat{\alpha}\subseteq \alpha $.
So in particular, $\alpha $ is active at $\beta ^{\prime }$. Now $\beta
^{\prime }$ has a witness $x^{\prime }$ necessarily less than $x$ (as $x$ is
appointed later) and at stage $v$ when we reached $\beta ^{\prime }\symbol{94%
}g_{\hat{\alpha}}$ we put $\gamma (x^{\prime })<\gamma (x)$ into $P$ and so
kill the current computation of $\Gamma (x)$ and correct it when we next
reach $\alpha \symbol{94}i$. Note that both $P$ and $Q$ are r.e. so $\gamma
(P;x)-1$ has never been in $P$ before and $x$ will never leave $Q$. As we
impose permanent restraint when we go to outcome $d$ and diagonalize, no
later change in $W$ can return us to any old computation.
\end{proof}

\begin{lemma}
\label{Delta}The functional $\Delta $ defined at the true $g_{\alpha }$
outcome of a $\Theta $ node $\beta $ (for $G_{i}$) on the true path starting
at $s(\beta \symbol{94}g_{\alpha })$ is well-defined. When $\Delta (u)$ is
convergent while we are at $\beta \symbol{94}g_{\alpha }$, it always give
the current value of $W(u)$ on an initial segment of $\omega $. Indeed, $%
\Delta (G_{i})=W$ as desired. (For notational convenience we assume that $%
\alpha $ is assigned the $\Phi $ requirement for $W$ which is constructing
the functional $\Gamma $ at its $i$ outcome.)
\end{lemma}

\begin{proof}
As for the final claim, note first that by construction if $\Delta (u)$ is
ever defined it is defined at every $u^{\prime }<u$ and then it is defined
there at every later $P$-stage at which $\beta \symbol{94}g_{\alpha }$ is
accessible. Moreover, at these stages we extend its domain of definition to $%
l_{\alpha }(s)$ which is going to infinity since $\alpha \symbol{94}%
i\subseteq \beta $ is on the true path. Once $\Delta (u)$ is defined, its
use never changes by construction and so if, at every stage when defined at $%
\beta \symbol{94}g_{\alpha }$, it correctly computes $W(u)$, it does so in
the end.

For correctness, we argue by induction on the stages at which $\beta \symbol{%
94}g_{\alpha }$ is accessible beginning at $s(\beta \symbol{94}g_{\alpha })$
that $\Delta (u)=W(u)$ at every $u$ at which $\Delta $ is defined. This is
obviously true by construction if $s$ is the first stage at which $\Delta
(u) $ is defined. Suppose it is true at $s$ and the next stage at which $%
\beta \symbol{94}g_{\alpha }$ is accessible is $t$ and the problem occurs at 
$u$.

Note that no change in $A\upharpoonright s$ or $P\upharpoonright \theta (x)$
can occur between $s$ and $t$. No node to the right of $\beta \symbol{94}%
g_{\alpha }$ can make such a change by Lemma \ref{after init}. No node to
its left can do so as they are never accessible after $s(\beta \symbol{94}%
g_{\alpha })$. No node above $\beta \symbol{94}g_{\alpha }$ can change $A$
at all by Lemma \ref{change A}. Nodes above $\beta \symbol{94}g_{\alpha }$
may act to put numbers of the form $\gamma _{\hat{\alpha}}(P;\hat{x})$ into $%
P$ via other $\Theta $ requirements $\hat{\beta}$ above $\beta $ with
outcome $g_{\hat{\alpha}}$ but by our believability condition on the $\Theta
(x)$ computation, at stage $s$ none of them are below $\theta (x)$ at $s$.
The only way one of them could decrease is if it had previously been
increased from $u_{1}$ to $u_{2}$ by a change in $W$ as described at the end
of \S \ref{t1s3c2} and then $W$ changes back to the old value before the old
computation is killed by $v_{1}-1$ going into $P$. However, our
believability condition requires that $\theta (x)$ is also less than these $%
v_{1}$.

If $s$ was a $P$-stage then no change occurred in $A\upharpoonright s$ at $s$
so none has by stage $t$. Thus if $W(u)$ at $t$ is different from its value
at $s$ it would be different from $\Phi (A;u)$ at $t$ since that is the same
as it was at $s$. This would move us to outcome $w$ at $\alpha $ and so $%
\beta \symbol{94}g_{\alpha }$ would not be accessible for a contradiction.

Suppose then that $s$ was an $A$-stage. No change in $P\upharpoonright
\theta (x)$ occurs at $s$ as no node above the one announcing the $A$-stage
can change $P$ without declaring a $P$-stage or moving left and none after
it can because it is an $A$-stage. Moreover, none can occur before $t$ as
above. If no change occurs in $A$ at $s$ then, as none occurs before $t$, $%
\Phi (u)$ and $\Delta (u)$ would be the same at $t$ as at $s$. If this value
is not that of $W(u)$ at $t$ then $\alpha \symbol{94}i$ would again not be
accessible at $t$ for a contradiction. Thus there has been some change in $A$
at $s$. By Lemma \ref{only one}, there is precisely one $z$ that entered or
left its block of sets at $s$. By Lemmas \ref{change A} and \ref{after init}%
, the node $\sigma $ causing the change must extend $\beta \symbol{94}%
g_{\alpha }$ as we are after $s(\beta \symbol{94}g_{\alpha })$ and $\beta 
\symbol{94}g_{\alpha }$ is accessible. Now if there is no change in $W(u)$
between $s$ and $t$, then the only way we could have a disagreement with $%
\Delta (u)$ at $t$ (so the old axiom for $\Delta (u)$ at $s$ is no longer
valid but we cannot simply put in a new one with the same value) is that the
change for $z$ returns us to a previous computation of $\Delta (u)$ giving a
different value. However, such a change in $z$ can be caused only by $\sigma 
$ shuffling $z$. Such a shuffle returns $A\upharpoonright v$ to its value at
a previous stage $v$. If $\Delta (u)$ was defined at $v$ then by induction
it would have the same value as $\Phi (u)$ and $W(u)$ and no change can have
happened in any of these when we reach $t$. Thus we would still have
agreement at $t$ as required. So we may also assume that $W(u)$ is different
at $s$ and $t$.

Suppose first that $z<\theta (x)$ and $G_{i}$ is in its block of sets. Next,
suppose the change occurred because of some shuffling procedure at $\sigma $%
. If $\Delta (u)$ was first defined before the shuffling began at $\sigma $,
then we would have a contradiction as above.

If $\Delta (u)$ was first defined after the shuffling began, say at $v\leq s$
with, for the sake of definiteness, $z\in G$, then its use is fresh at $v$
and so larger than $z$. Let $v^{\prime }\geq v$ be the next stage after $v$
at which we shuffle $z$ at $\sigma $. If $v^{\prime }=s$ then $z$ has been
in $G_{i}$ from $v$ to $s$ and is removed at stage $s$ by our action at $%
\sigma \supset \beta \symbol{94}g_{\alpha }$. Thus when we next return to $%
\beta \symbol{94}g_{\alpha }$ at $t$ we have $z\notin G_{i}$ for the first
time since $\Delta (u)$ was defined and we redefine it to be the current
value of $W(u)$ as required. So we may assume that $v^{\prime }<s$ and at
stage $v^{\prime }$ we have $\Phi (u)=W(u)=\Delta (u)$ at $\beta \symbol{94}%
g_{\alpha }$ by induction. When we reach $\sigma $ at $v^{\prime }$ we
remove $z$ from its block of sets including $G_{i}$ and impose permanent
restraint. When $\beta \symbol{94}g_{\alpha }$ is next accessible, say at $%
v^{\prime \prime }\leq s$, we redefine $\Delta (u)=W(u)=\Phi (u)$ with $%
z\notin G_{i}$ (and its block of sets) but with the rest of $%
A\upharpoonright v^{\prime }$ the same as it was at $v^{\prime }$ (because
of the permanent restraint imposed at $v^{\prime }$ which, by Lemma \ref%
{change A}, could be violated only by moving to the left of $\sigma $ which
could then not cause our problem at $s)$. If $v^{\prime \prime }=s$ then at
stage $s$ we shuffle $z$ back into $G_{i}$ (and its block of sets) at $%
\sigma $ and impose permanent restraint. We next return to $\beta \symbol{94}%
g_{a}$ at $t$ and have $\Phi (u)$ and $\Delta (u)$ and so $W(u)$ the same as
they were at stage $v^{\prime }$ at $\beta \symbol{94}g_{\alpha }$, i.e.
they all agree as required. Finally, if $v^{\prime \prime }<s$ then later at
stage $s$ we shuffle at $\sigma $ between the values for all of these sets
and functionals that we had at $v^{\prime }$ and $v^{\prime \prime }$. Thus
once again when we reach $\beta \symbol{94}g_{\alpha }$ at $t$ all agree.

Thus the change that has occurred is that $z$ entered $G_{i}$ for the first
time at $s$. If the change in $W(u)$ is not that $u$ has entered for the
first time, then we would have initiated a shuffling procedure at $\alpha $
at stage $t$ and so move to $\alpha \symbol{94}s_{1}$ hence to the left of $%
\beta \symbol{94}g_{\alpha }\supset \alpha \symbol{94}i$ for a contradiction.

Thus through stage $s$, $W(u)=0$. Suppose $\Delta (u)$ was first defined at $%
s^{\prime }$, of course with value $0$ and fresh use $q$. We claim that $z<q$
and so its entry into $G$ allows us to correct $\Delta (u)$ at $t$ as
desired. If not, it was chosen fresh as a witness for $\sigma $ at a point
in the construction during a stage $s^{\prime \prime }\geq s^{\prime }$
after $\Delta (u)$ was defined at $s^{\prime }$ but before $s$. In this
case, however, $z>s^{\prime \prime }$ and so $z>\phi (u)$ at $s^{\prime
\prime }$. Note that $\Phi (u)$ is defined at $s^{\prime \prime }$ because $%
z $ is appointed at $\sigma \supset \beta \supset \alpha \symbol{94}i$. Now
from the point of stage $s^{\prime \prime }$ at which $\alpha \symbol{94}i$
is accessible to stage $s$ any change in $A\upharpoonright s^{\prime \prime
} $ would initialize $\sigma $ and so $z$ could not enter $A$ at $s$. (At $%
s^{\prime \prime }$ no node between $\alpha \symbol{94}i$ and $\sigma $ can
change $A$ without moving left of $\sigma $. Then at $\sigma $ all nodes to
the right of $\sigma $ are initialized and so cannot make any changes below $%
s^{\prime \prime }$ by Lemma \ref{after init}. As $\sigma $ appointed a
witness at $s^{\prime \prime }$, this is the first stage at which $\sigma $
has been accessible since it was last initialized so all $A$ action by nodes
below $\sigma $ also involve only numbers larger than $s^{\prime \prime }$.
Finally, any $A$ action after $s^{\prime \prime }$ by a node of higher
priority than $\sigma $ would also initialize it by Lemma \ref{change A}.)
Thus at $s$, $\phi (u)$ and $A\upharpoonright \phi (u)$ are the same as they
were at $s^{\prime \prime }$, i.e. $\Phi (A,u)=0=W(u)$ at $s$ with the same
computations as at $s^{\prime \prime }$. Now the only changes in $A$ during
stage $s$ is that $z$ enters its block of sets but $z>s^{\prime \prime }$
and then no changes occur in $A\upharpoonright s$ before stage $t$. Thus at
stage $t$ we also have $\Phi (A,u)=0$ and so if $W(u)=1$ at $t$, the outcome
of $\alpha $ would not be $i$, for a contradiction.

Finally, suppose $z\geq \theta (x)$ or $G_{i}$ is not in its block so no
change occurs in $G_{i}\upharpoonright \theta (x)$ at $s$ and so none before 
$t$. When $\Delta (u)$ was defined at the $P$-stage $v$ (necessarily before
the $A$-stage $s$), $u<l_{\alpha }(v)$ and so after $v$, $u<\gamma (W;x)$
whenever it is defined. In fact, at each $P$-stage during which we reach $%
\beta \symbol{94}g_{\alpha }$ (starting with $v$) we put $\gamma (P;x)$ into 
$P$ and subsequently increase $\gamma (W;x)$ to $l_{\alpha }(v^{\prime
})>l_{\alpha }(v)$ when we are next at $\alpha \symbol{94}i$ (at $v^{\prime
}>v$). Each computation of $\Gamma (x)$ killed in this way can never to
reapply to $W\oplus P$ as $P$ is r.e. As we cannot reach $\beta \symbol{94}d$%
, the only other way $\gamma (x)$ can change requires a $W$ change that
causes a difference between the previously computed common values of $\Delta 
$ and $W$. By our induction assumption this cannot have occurred before $s$.
So all axioms for $\Gamma (x)$ provided before $s$ are invalid by the end of
stage $s$.

As $W(u)$ has different values at $s$ and $t$, the change in $W$ introduces
a value of $W(u)$ that we see at $\alpha \symbol{94}i$ at some first stage $%
v^{\prime \prime }$ after $s$ but no later than $t$. As we have argued, no
old computation of $\Gamma (x)$ is still valid at $v^{\prime \prime }$. Thus
by construction we would increase $\gamma (P;x)$ at $v^{\prime \prime }$ to
a fresh value larger than $\theta (x)$. When we return to $\beta \symbol{94}%
g_{\alpha }$ at $t$, $\gamma (P;x)$ is now larger than $\theta (x)$ which
has not changed since $s$. Thus by construction, $g_{\alpha }$ cannot be the
outcome of $\beta $ at $t$ for a contradiction.
\end{proof}

\subsection{All requirements are satisfied}

\label{t1s4ss5}

Finally we want to show that all requirements are satisfied.

The positive order requirements are easily verified by our construction.

\begin{lemma}
\label{pos order}If $i<_{\ast }j$ then $G_{i}\leq _{T}L\oplus G_{j}$.
\end{lemma}

\begin{proof}
Consider an $x>i,j$. To decide if $x\in G_{i}$ go to stage $x$ of the
construction and see if $x$ has been appointed as a witness for some $\Pi $
or $\Psi $ requirement with $G_{i}$ in its block. If not, then $x\notin
G_{i} $. (Indeed $x$ is not in any $G_{k}$.) If it is in the block for a $%
\Pi $ requirement then $L$ is also in its block. If for a $\Psi $
requirement then $G_{j}$ is in the block. In any case, as once appointed $x$
moves into or out of all sets in its block during the entire construction, $%
x\in G_{i}\Leftrightarrow x\in L$ in the $\Pi $ case and $x\in
G_{i}\Leftrightarrow x\in G_{j}$ in the $\Psi $ case.
\end{proof}

We now move to the negative (diagonalization) requirements.

\begin{lemma}
\label{Pi or Psi} The $\Pi $ and $\Psi $ requirements are satisfied.
\end{lemma}

\begin{proof}
Suppose the requirement is assigned to the node $\beta $ on the true path.
If, after $s(\beta )$, we ever go to outcome $d$ and so diagonalize, the
result is immediate from the construction and Lemma \ref{rest A}. If not, it
must be that the outcome is always $w$ after after $s(\beta )$. If the
relevant computation converged to $0$ the correct computation would be
available from some point on and so by Lemma \ref{alt} we would eventually
see it at an $A$-stage and so move to outcome $d$ by Lemma \ref{wit large}.
If not, then $x$ never enters $G_{i}$ and we also satisfy the requirement as
desired.
\end{proof}

We next consider the $\Theta $ requirements.

\begin{lemma}
\label{Theta d}If a $\Theta $ node $\beta $ on the true path has true
outcome $d$ then the associated requirement is satisfied.
\end{lemma}

\begin{proof}
Consider the stage $s(\beta \symbol{94}d)$ when $\beta $ has outcome $d$
(and is never again initialized). We put the witness $x$ into $Q$ and impose
permanent restraint to preserve the computations $\Theta (G_{i}\oplus P;x)=0$%
. Lemma \ref{rest A} shows that this computation is preserved.
\end{proof}

\begin{lemma}
\label{Theta g}\label{t1s4ss5sss3}If a $\Theta $ node $\beta $ on the true
path has true outcome $g_{\alpha }$, then its requirement is satisfied.
Indeed, for $x$ the final witness for $\beta $, both $\theta (x)$ and $%
\gamma _{\alpha }(x)$ go to infinity on the stages when $\beta \symbol{94}%
g_{\alpha }$ is accessible. Moreover, any time we increase $\gamma _{\alpha
}(x)$ because of a $W$ change as at the end of \S \ref{t1s3c2}, we later
kill the $P$-use of the old computation as well by putting $v_{1}$ into $P$.
\end{lemma}

\begin{proof}
In this case, by our construction (and Lemmas \ref{wit large} and \ref{alt}%
), we infinitely often put numbers ($\gamma (P;x)-1$) into $P$ and redefine $%
\Gamma (W_{i}\oplus P)$ with a fresh $P$-use. (The first of these Lemmas
implies that the numbers we want to put into $P$ are larger than any
permanent restraint as they are of the form $\gamma (P;x)$ which is larger
than the witness $x$ for $\beta $.) So by our criteria for going to outcome $%
g_{\alpha }$, we infinitely often see $\theta (x)>\gamma (P;x)$, so $\theta
(x)$ must go to infinity (when $\beta \symbol{94}g_{\alpha }$ is accessible)
along with $\gamma (P;x)$ and the computation $\Theta (G_{i}\oplus P;x)$
diverges. As for any increase in $\gamma (x)$ because of $W$ change as in \S %
\ref{t1s3c2}, the next time we are at $\beta \symbol{94}g_{\alpha }$ we put
the associated $v_{1}$ into $P$ by construction.
\end{proof}

\begin{lemma}
\label{Theta w}If a $\Theta $ node $\beta $ on the true path has true
outcome $w$ then the associated requirement is satisfied.
\end{lemma}

\begin{proof}
Note that if the outcome of $\beta $ were $d$ at any stage after $s(\beta )$
then $d$ would be the true outcome by construction. Thus in our case, we
never put the final witness $x$ for $\beta $ into $G_{i}$. So our only
concern is that $\Theta (G_{i}\oplus P;x)=0$. In this case, there is a stage
after which it always converges to $0$ and with a fixed use. By the previous
Lemma this computation is believable at almost every stage when we are at $%
\beta $. Thus by construction and Lemmas \ref{wit large} and \ref{alt}, as
in Lemma \ref{Theta g}, we would eventually have outcome $d$ for a
contradiction.
\end{proof}

We now turn to the $\Phi $ requirements.

\begin{lemma}
\label{Phi s} If a $\Phi $ node $\beta $ on the true path has true outcome
of type $s$, then the associated requirement is satisfied.
\end{lemma}

\begin{proof}
The nature of the shuffling points guarantees that, at every stage with
outcome of type $s$, $\Phi (A;x)\downarrow \neq W(x)$ and so this is true at
the end of the construction as well and the requirement is satisfied. The
crucial point here is that the permanent restraint imposed by $\beta $ which
are increasing as we move left among the type $s$ outcomes can never be
injured (other than by the shuffling done by $\beta $ itself) by Lemma \ref%
{rest A}.
\end{proof}

\begin{lemma}
\label{Phi w}If a $\Phi $ node $\beta $ on the true path has true true
outcome $w$ then the associated requirement is satisfied.
\end{lemma}

\begin{proof}
If $\Phi (A)=W$ then the length of agreement would go to infinity and so, by
construction and Lemma \ref{alt}, we would eventually move to outcome $i$
after $s(\beta \symbol{94}w)$ for a contradiction.
\end{proof}

Finally, we have to deal with the case that every $\Phi $ node on the true
path has true outcome $i$.

\begin{lemma}
\label{Phi i}\label{t1s4ss5sss2}Every $\Phi $ requirement is satisfied.
\end{lemma}

\begin{proof}
As usual in a $0^{\prime \prime \prime }$ priority tree argument, we want to
consider the last node $\alpha $ along the true path assigned to a given $%
\Phi $ requirement. To see that there is such a node, argue by induction on
the $\Phi $ requirements. The point here is that any $\Phi $ requirement,
once assigned to a node that is never again initialized, can return to the
list of requirements from which we draw to make assignments of requirements
to nodes (along the true path) only when a strictly higher priority node
becomes inactive. So once no node with a higher priority $\Phi $ requirement
assigned ever becomes inactive again, the next node $\beta $ assigned to $%
\Phi $ either becomes inactive once along the true path (by being satisfied
by action at a lower $\Theta $ node on the true path) and then remains
inactive or it never becomes inactive on the true path. In either case, $%
\Phi $ is never assigned to a node below $\beta $ by the definition of the
priority tree.

Let $\alpha $ be the last node along the true path assigned to the $\Phi $
requirement ($\Phi (A)=W_{i}$). By the previous two Lemmas, we may assume
that its true outcome is $i$. If there is an $\Theta $ node $\beta $ ($%
\Theta (G_{k}\oplus P)\neq Q$) on the true path with true outcome $g_{\alpha
}$, then we have built a functional $\Delta $ at $\beta \symbol{94}g_{\alpha
}$ that computes $W_{i}$ from $G_{k}$ by Lemma \ref{Delta}.

If there is no such $\Theta $ node $\beta $, then we claim that we have
successfully built $\Gamma (W_{i}\oplus P)=Q$ starting at $s(\alpha \symbol{%
94}i)$. By Lemma \ref{Gamma}, we only have to verify that $\gamma (x)$ is
eventually constant for each $x$. We begin to define our $\Gamma $ at $%
s(\alpha \symbol{94}i)$. Assume inductively that $\gamma (\hat{x})$ has
stabilized for $\hat{x}<x$. Thereafter, once $\Gamma (x)$ is defined, our
construction allows $\gamma (x)$ to increase because of a change in $P$ at
most once for each time some $\beta \symbol{94}g_{\alpha }$ below $\alpha 
\symbol{94}i$ is accessible and the $\Theta $ requirement assigned to $\beta 
$ has witness $x$ or once when $\beta \symbol{94}d$ is accessible (again
with witness $x$ for $\beta $). It can increase because of a $W$ change at
most finitely often for each such $\beta \symbol{94}g_{\alpha }$ and stage.
(At worst only when $W$ changes on the domain of $\Delta $ at that stage.)
At most one $\beta $ has $x$ assigned as a witness. If $\beta $ is not on
the true path, it can be accessible with witness $x$ at most finitely often.
If $\beta $ is on the true path, once $\beta \symbol{94}d$ is accessible, $%
\beta \symbol{94}g_{\alpha }$ cannot be accessible again unless $\beta $ is
initialized and so chooses a new witness. If $\beta \symbol{94}g_{\alpha }$
is accessible infinitely often, then some $g_{\hat{\alpha}}$ (possibly to
the left of $g_{\alpha }$) would be its true outcome and so $\hat{\alpha}%
\subseteq \alpha $. If $\alpha =\hat{\alpha}$ we contradict our case
assumption. If $\hat{\alpha}\subset \alpha $ then $\alpha $ would be come
inactive and $\Phi $ would be reassigned later to a node below $\beta 
\symbol{94}g_{\alpha }$ on the true path contradicting our choice of $\alpha 
$. Thus $\Gamma (x)$ can change at most finitely often as required.
\end{proof}

\subsection{$\Delta _{2}^{0}$ and $\Delta _{3}^{0}$ partial orders\label{0'
0''}}

To handle partial orders recursive in $0^{\prime }$ we make the following
changes in the construction:

We begin with a recursive approximation $f(i,j,s)$ to the (characteristic
function of the) relation $i\leq _{\ast }j$. We now have requirements $\Psi
_{e,i,j}$ for every $e,i,j$ with a new additional leftmost outcome $n$. At
stage $s$ at a node $\alpha $ for $\Psi _{e,i,j}$, if $f(i,j,,s)=1$ (so we
think we do not want to diagonalize) we go to outcome $n$ and do nothing. If 
$f(i,j,s)=0$ we act as before with a new definition of the block for our
witness $x$. When $x$ (necessarily larger than $i$ and $j$) is appointed as
a witness, we determine its block by calculating $f(k,l,t)$ for each $k,l<x$
and $t>x$ until we reach a $t$ at which either $f(i,j,t)=1$ or the relation
on numbers $k,l<x$ defined by $f(k,l,t)$ is a partial order $\preceq $. In
the first case, the outcome is again $n$ and we do nothing. In the second
case, we put $G_{k}$ into the block for $x$ if and only if $i\preceq k$
(i.e. $f(i,k,t)=1$). Note that $f(i,j,t)=0$ by our case assumption and so $%
G_{j}$ is not in the block.

To see that this modification works, note that if $i\nleq _{\ast }j$ then
from some point on $f(i,j,t)=0$ and so we never again have an outcome $n$
for $\Psi _{e,i,j}$ and so satisfy the negative order requirements as
before. For the positive ones, suppose $k\leq _{\ast }l$ and for $t\geq
t_{0} $, $f(k,l,t)=1$. For any witness $x\geq k,l,t_{0}$, if its block does
not contain $k$ then, of course, $x\notin k$. If it does contain $k$ it also
contains $l$ and so $x\in G_{k}$ if and only if $x\in G_{l}$. (The case for $%
\Pi $ requirements is as before.)

The modifications needed for partial orders recursive in $0^{\prime \prime }$
are more complicated. For each $i,j$ we insert a requirement into the
priority order used for the $\Delta _{2}^{0}$ case and so on each path of
the priority tree a node $\varepsilon $ that guesses in a $\Delta _{3}^{0}$
way whether $i\leq _{\ast }j$, i.e. the node has infinitely many outcomes $%
\left\langle x,k\right\rangle $ with $x\in w$ and $k\in \{0,1\}$ ordered
lexicographically. We organize determining the outcome of $\varepsilon $ at
each stage $s$ so that if $\left\langle x,k\right\rangle $ is the leftmost
outcome accessible infinitely often then $x$ is the least witness to the $%
\Sigma _{3}$ formula which says that $i\leq _{\ast }j$ if $k=1$ and the
least witness to the $\Sigma _{3}$ formula which says that $i\nleq _{\ast }j$
if $k=0$. In addition we coordinate this guessing with the stage
announcements so that the true outcome passes on alternating restraint as
before. We then act at nodes as in the $\Delta _{2}^{0}$ case but using at
each node only the information about the ordering coded on the outcomes of
the $\varepsilon $ type nodes above it. So for for a $\Psi $ type
requirement, if the relation given in this way is not a partial order $%
\preceq $ or says that $i\preceq j$, we go to outcome $n$. (We put the $%
\varepsilon $ nodes on the tree so that any node for a requirement $\Psi
_{e,i,j}$ has an $\varepsilon $ type node above it assigned to $i\leq _{\ast
}j$.) If it is does specify a partial order with $i\npreceq j$, then we act
as before but now the block of sets for a witness $x$ consists of all $G_{k}$
with $i\preceq k$. One can now verify that the construction works. The
argument for the positive order relations runs as follows: If $i\leq _{\ast
}j$, find the node $\sigma $ on the true path by which that fact has been
decided. Nodes to the left of $\sigma $ put only finitely many $x$ into $%
G_{i}$ and can be ignored. For nodes to its right that appoint any witness $%
x $ (necessarily before stage $x$), we can wait for the node to be
initialized to see if $x$ enters $G_{i}$. For nodes below $\sigma $ in the
tree assigned to any $\Psi $ requirement, any witness $x$ that puts $G_{i}$
in its block also puts $G_{j}$ and so for those $x$, $x\in
G_{i}\Leftrightarrow x\in G_{j} $. Of course, for witness $x$ for $\Pi
_{e,i,k}$ type nodes, $x\in G_{i}\Leftrightarrow x\in L$ as before. Of
course, for any $x$ not appointed as a witness for one of these type nodes, $%
x\notin G_{i}$.

If the partial order is only $\Sigma _{3}$, then one adjust the previous
procedure by instead of single nodes with $\Delta _{3}$ guessing at $i\leq
_{\ast }j$ putting in individual nodes for each $i$ and $j$ guessing that a
particular number is the (least) witness to the $\Sigma _{3}$ fact that $%
i\leq _{\ast }j$. Along a path with the $\Pi _{2}^{0}$ outcome that the
witness is correct, one follows a coding stratgey incorporating this
individual fact. If it is true, then some node $\sigma $ on the true path
has the correct witness and all nodes below it obey the required coding
strategy. Nodes not below this one, are handled as above. For each node
guessing a witness for the $\Sigma _{3}$ fact, where we see that it is
false, i.e. along a path with the $\Sigma _{2}^{0}$ outcome, we put in one
more $\Psi $ requirement for $i\nleq _{\ast }j$. So if $i\nleq _{\ast }j,$
then along the true path, we will put in $\Psi _{e,i,j}$ requirements for
every $e$ and so satisfy the requirement.

\section{Requirements II\label{Req II}}

We now turn to our second technical result: \setcounter{section}{1} %
\setcounter{theorem}{6}

\begin{theorem}
For any $n\geq 1$, there are r.e. degrees $\mathbf{g},\mathbf{p},\mathbf{q}$%
, an $n$-r.e.\ degree $\mathbf{a}$ and an $n+1$-r.e.\ degree $\mathbf{d}$
such that:

\begin{enumerate}
\item For every $n$-r.e.\ degree $\mathbf{w}\leq \mathbf{a}$, either $%
\mathbf{q}\leq \mathbf{w}\vee \mathbf{p}$, or $\mathbf{w}\leq \mathbf{g}$.

\item $\mathbf{d}\leq \mathbf{a}$, $\mathbf{q}\nleq \mathbf{d}\vee \mathbf{p}
$, and $\mathbf{d}\nleq \mathbf{g}$.
\end{enumerate}
\end{theorem}

\setcounter{section}{7} \setcounter{theorem}{0} Our list of requirements is
very similar to the one used for our first technical theorem:

\begin{enumerate}
\item $\Psi_e: \Psi_e(G)\neq D$;

\item $\Theta_e: \Theta_e(D\oplus P)\neq Q$;

\item $\Phi_{e,i}: (\Phi_e(A)=W_i)\rightarrow [ \exists\Gamma(
\Gamma(W_i\oplus P)=Q) \vee \exists\Delta (\Delta(G)=W_i)]$.
\end{enumerate}

In addition, we need to make $D\leq _{T}A$. Note that we only add elements
into $D$ by diagonalization for $\Psi $ requirements. Whenever we pick a
witness $x$ for $D$, $x$ is fresh at that stage, and we reserve the pair $%
(x,x+1)$ for coding $D$ into $A$. If $x$ enters $D$ for the first time, then
we also put $x$ into $A$. If $x$ leaves $D$ later, we either take $x$ out of 
$A$ or put $x+1$ into $A$. In the first case, we may shuffle $x$ into and
out of $A$ and $D$ simultaneously but allowing at most $n$ changes. In the
second case, we may shuffle $x+1$ into and out of $A$ and $D$ simultaneously
again allowing at most $n$ changes in $A$ (but this may make for $n+1$
changes in $D$ altogether). Therefore in the end $x$ is in $D$ if and only
if $x$ is in $A$ and $x+1$ is not in $A$. No numbers other than these $\Psi $%
-witnesses enter or leave $D$ in our construction, and so $D$ is recursive
in $A$, $A$ is $n$-r.e. and $D$ is $(n+1)$-r.e.

\section{Priority Tree II\label{tree II}}

Our priority tree here is almost the same as the one used in the first
theorem. Of course, we do not have $\Pi $ nodes. For any $\Theta $ node $%
\beta $, we put a new [temporary] outcome $r_{\alpha _{j}}$ to the left of
each $g_{\alpha _{j}}$. So the outcomes of $\beta $ are $d$, $r_{\alpha
_{1}} $, $g_{\alpha _{1}}$, $r_{\alpha _{2}}$,...,$g_{\alpha _{k}}$ and $w$.
We do not add nodes below these type $r$ outcomes. [So a stage $s$ may
terminate at such an outcome before we reach level $s$ of the priority tree.
We show, however, in Lemma \ref{Plan R lem} that no node of type $r$ can be
on the true path.] The notions of active $\Phi $ nodes, $\alpha -\beta $
pairs are defined in the same way as in Section \ref{tree}.

\section{Construction II\label{con II}}

We only specify the construction at stage $s$ when there is no stage
announcement. In the case when there is a stage announcement, we act as in 
\S \ref{t1s3c4}. Note that in this construction we only change $G$ during $P$
stages. The default permanent restraint is on $A$, $D$ and $G$ while for $P$
it must be specifically mentioned.

\subsection{$\Psi$ node and $\Phi$ node}

At a $\Psi $ node, the action is the same as the one in \S \ref{t1s3c1} with 
$G$ for $L\oplus G_{i}$ and $D$ for $G_{i}$.

At a $\Phi $ node $\alpha $, we follow almost the same procedure as we did
in \S \ref{t1s3c2}. The only difference is in how we revise the computations
from old $\Gamma $-axioms. As in the first construction, if $P$ has changed
and the change was caused by some $\beta $ with $g_{\alpha }$ outcome by
putting the old use into $P$, then we increase the $W$-use to $l_{s}(\alpha
) $ and $P$-use to be fresh. If a $W$ change caused some $\Gamma (x)$ to be
undefined, then we check whether $x$ is a diagonalization witness for some $%
\beta $ below $\alpha $, if so, we also check whether $D$ has changed by
putting in some number for the first time at the previous stage when $\beta $
was accessible (and $\beta $ has not been initialized since). If so, we then
redefine $\Gamma (x)$ with $W$-use up to $l_{s}(\alpha )$ and fresh $P$-use.
In all other cases we redefine the axiom without changing the uses.

\subsection{$\Theta$ node}

[At a $\Theta $ node, the obvious difference from the first construction is
that we use $D$ in our $\Theta $ computation but use $G$ in our $\Delta $
computation. So the arguments in \ref{t1s4ss3} are no longer valid. In the
case that $W$ changes, we have no reason to expect a $G$ change. In fact, so
far we have no requirements or procedures that put numbers into $G$. Here we
actively put numbers into $G$ to correct $\Delta $ computations. We will
make full use of the fact that $D$ is $n+1$-r.e., i.e., it has one more
chance to change than $A$ and the $W_{i}$. We may remove a number from $D$
while leaving it in $A$ but putting $z+1$ into $A$. This will afford us the
opportunity to produce a situation in which we may initiate shuffling.]

At a $\Theta $ node $\beta $ accessible for first time after it has been
last initialized, we pick a fresh witness $x$ for diagonalizing $\Theta
(D\oplus P)\neq Q$. If we have a witness $x$ already assigned (and not yet
canceled by initialization) at $\beta $, we check whether the $\Theta $
computation converges at the witness $x$. If we do not have a believable
(defined in the same way as in our first theorem) computation $\Theta
(D\oplus P;x)$, we go to outcome $w$. If we do, we follow Plan D as in \S %
\ref{t1s3c3ss1} if we can. If not, we have a planned outcome $g_{\alpha
_{j}} $ as in \S \ref{t1s3c3ss2} and check whether we have not been at this
outcome since $\beta $ was last initialized or whether the previous stage $t$
when we went to this outcome was a $P$-stage. If so we announce an $A$-stage
and continue the construction below $\beta $.

Otherwise, we have two possibilities.

\subsubsection{Plan R: removal\label{Plan R}}

Let $t$ be the last stage at which $\beta $ was accessible. If there is a $%
y<\gamma (W;x)$ such that $W(y)$ has different values at $t$ and $s$ and the
only change in $A\upharpoonright t$ is that some element $z$ entered $D$ and 
$A$ for the first time at stage $t$ because of the action of a node below $%
\beta $ [necessarily a $\Psi $ node], we remove $z$ from $D$ and add $z+1$
into $A$ [so we restore the version of $D$ at stage $t$ up to the $\theta $
use]. We go to the $r_{\alpha _{j}}$ outcome and terminate the current stage
of the construction. We call the least such $y$ the \emph{key witness} for
the removal plan.

[The idea here is that, before $\beta $ can become accessible again without
being initialized, we would see at $\alpha _{j}$ if $y$ has left $W$. If so
we would initiate a shuffle there on $z+1$ and initialize $\beta $. If not,
we will argue that we must go to the left of $g_{\alpha _{j}}$ and $%
r_{\alpha _{j}}$. Roughly, the idea is that the computation $\Theta (D\oplus
P;x)$ will be the same as that at stage $t$ while $\gamma _{j}(P;x)$ will
have been increased above $\theta (x)$ by $y$ entering $W$. ]

\subsubsection{Plan G: change G\label{Plan G}}

If we satisfy none of the above criteria, we go to the outcome $g_{\alpha
_{j}}$ and continue to build $\Delta $ consistently. For each $u$, if $%
\Delta (u)$ was defined at the last stage $t$ at which $\beta \symbol{94}%
g_{\alpha _{j}}$ was accessible and $W(u)$ has not changed since then, we
simply update the $\Delta $ axiom with the current version of $G$ (if
necessary) without changing the use. If $\Delta (u)$ was defined at $t$ but $%
W(u)$ is now different, then let $\delta (u)$ be the use of $G$ in the old $%
\Delta $ computation. We add $\delta (x)-1$ into $G$ and redefine $\Delta $
with a fresh use in $G$. [We preserve the consistency of $\Delta $ by doing
this as $G$ is r.e.] Then we also define a new computation $\Delta (u)$ for
the next $u$ which was undefined with fresh $G$-use. Finally, we follow Plan
P to add $\gamma $ uses into $P$ as in Section \ref{t1s3c3ss3} and announce
a $P$-stage.

\section{Verification II\label{ver II}}

We can go through most of Section \ref{t1s5} and show that we have a
leftmost path visited infinitely often (that it is actually infinite follows
from Lemma \ref{Plan R lem}), and each node along the true path is passing
down alternating $A$-stages and $P$-stages along the true path. There are
obvious alphabetic changes needed In Lemmas \ref{change A}-\ref{alt} -- no $%
L $ or $\Pi $. Otherwise, note first that Plan R action for $\Psi $ nodes
are an exception to Lemma \ref{change A}. Next, Lemma \ref{only one} applies
to $D$ as well as $A$ and we have to remark that if we implement Plan R no
node is even accessible thereafter, while Plan R action cannot be the second
type to change $A$ (or $D$) at $s$ by the arguments given in the proof of
Lemma \ref{only one}. Finally, for the proof of \ref{after init} note that $%
G $-uses for $\Delta $ are also chosen fresh. It is then not difficult to
see that functionals are well-defined and complete the job we assigned if
they are along the true path. For the $\Gamma $'s, use Lemma \ref{Gamma}.
For the $\Delta $'s, it is directly guaranteed by our construction. [The
complicated argument for Lemma \ref{Delta} is not needed but see the proof
of Lemma \ref{Plan R lem} for some remnants of it.]

For the verification that all the requirements are satisfied we continue as
in \S \ref{t1s4ss5}. The positive order requirements (Lemma \ref{pos order})
are simply replaced by the requirement that $D\leq _{T}A$. Our construction
guarantees that $x$ is in $D$ if and only if $x$ is in $A$ and $x+1$ is not.
Except for the satisfaction of the $\Psi $ requirements (Lemma \ref{Pi or
Psi}) in the case of $d$ outcome all the other verifications proceed as in 
\S \ref{t1s4ss5}.

As for the $\Psi $ requirements, the major issue is that here we add
elements into $G$ (when we construct $\Delta $) and this action might, \emph{%
a priori}, injure some apparently satisfied $\Psi $ requirement of lower
priority. To show that the $\Psi $ requirements are all satisfied, we first
need a few lemmas.

\begin{lemma}
\label{Plan R lem} No outcome of type $r$ can be on the true path.
\end{lemma}

\begin{proof}
The argument is similar to the one in the end of the proof of Lemma \ref%
{Delta}. Consider any $\Theta $ node $\beta $ on the true path and suppose,
for the sake of a contradiction, that its true outcome is $r_{\alpha }$.

Let $s=s(\beta \symbol{94}r_{\alpha })$ and, as in the construction, let $t$
be the previous stage at which $\beta $ had outcome $g_{\alpha }$; $x$, the
diagonalization witness at $\beta $; $y$, the key witness for Plan R at $s$;
and $z$, the unique element that entered $D$ for the first time at $t$. We
remove $z$ from $D$ at $s$ following Plan R. Let $s^{\prime }$ be the next
stage at which $\beta $ is accessible with a believable $\Theta $
computation. If $y$ was ever out of $W$ between $s$ and $s^{\prime }$ (when $%
\alpha $ was accessible) then, we would have initiated a shuffle plan at $%
\alpha $ and so moved left of $\beta \supseteq \alpha \symbol{94}i$ for a
contradiction. (As we terminate stage $s$ at $\beta \symbol{94}r_{\alpha }$
with no action, there is no change in $A$ at $s$. Between $s$ and $s^{\prime
}$ Lemma \ref{after init} shows that $A\upharpoonright s$ is preserved. So
if $W(y)$ changes we satisfy the conditions for shuffling at $\alpha $.)
Thus we also assume that $y$ remains in $W$ at every stage at which $\alpha $
is accessible through stage $s^{\prime }$.

Now at stage $s$ we restored the computation $\Theta (D\oplus P,x)$ of stage 
$t$ by removing $z$ from $D$: $z$ is the only change to $D$ up to $\theta
(x) $ at $t$ by Lemma \ref{only one}, and $P$ is preserved by the $A$-stage
announcement at $t$. Between $t$ and $s$, $D\upharpoonright t$ is preserved
by Lemma \ref{after init} and $P\upharpoonright \theta (x)$ is not injured
by nodes to the right. Finally, our believability condition guarantees that $%
P\upharpoonright \theta (x)$ is also not injured by nodes above $\beta $.
Thus at stage $s^{\prime }$ we still have the same computation of $\Theta
(x) $ as at stage $t$.

Now consider the $\Gamma $ computation we build at $\alpha \symbol{94}i$. At
stage $s$ the conditions for Plan R guarantee that $y<\gamma (W;x)$ (at $t$)
has entered $W$ for the first time after $t$ and by $s$. So when this
happens and we are at $\alpha $ we see a change in the $W$ part of $\Gamma
(x)$ which makes $\Gamma (x)$ undefined. Therefore, by our construction, we
add a new axiom with $W$-use the current length of agreement and $P$-use
fresh, which is larger than $\theta (P,x)$ at $t$. This $\gamma (P,x)$
remains large since $y$ remains in $W$, therefore at stage $s^{\prime }$ we
will see that $\gamma (P,x)>\theta (P,x)$.

For other active $\alpha _{l}$'s between $\alpha $ and $\beta $, the
corresponding $\gamma _{l}(W_{l};x)$are larger than $\theta (P,x)$ at stage $%
t$ by the rules for going to outcome $g_{\alpha }$. The only way that anyone
of these uses can decrease is by $W_{l}$ changing at some stage $v\leq
s^{\prime }$ from its value on $\gamma _{l}(W_{l};x)$ at $t$ back to an
older version. If this happens, then when $\alpha _{l}$ is accessible at $%
v\leq s^{\prime }$ we initiate a shuffle at $\alpha _{l}$ by shuffling $z+1$
which we added into $A$ by Plan R at $t$ with shuffle points $t$ and $v$.
This would move us to the left of $\beta $ for a contradiction. Therefore
these uses cannot decrease, and at stage $s^{\prime }$ they are still larger
than $\theta (P,x)$. Hence at stage $s^{\prime }$ we will go to the left of
the $r_{\alpha }$ outcome by our construction for the desired contradiction.
\end{proof}

\begin{lemma}
\label{change G lem} In the construction at a $\Theta $ node $\beta $, if we
change $G$ as in \ref{Plan G} at a stage $s>s(\beta )$, then it must be the
case that at the previous stage $t$ when $\beta \symbol{94}g_{\alpha }$ was
accessible, we followed either Plan S or Plan R to change D at some node $%
\sigma $ below $\beta $.
\end{lemma}

\begin{proof}
By our construction, if we must change $G$ by putting in some $\delta (u)-1$%
, then $W(u)$ and therefore $A\upharpoonright \phi (u)$ must be different at
stage $s$ from stage $t$. By Lemmas \ref{after init} (for nodes to the
right), \ref{only one} (for nodes below), \ref{change A} (for nodes above
not of type $r$) and \ref{Plan R lem} (to see that there are no type $r$
nodes above), such a change in $A$ can only happen at stage $t$. Thus we
must have changed $A$ and $D$ at stage $t$ below $\beta $ since $\beta $ was
accessible.

We can change $A$ and $D$ in only three ways: diagonalization at a $\Psi $
node, Plan S or Plan R. By Lemma \ref{only one} if we have applied
diagonalization at a node below $\beta $, then it is the only change at
stage $t$, so at stage $s$ according to our construction we would want to
apply Plan R to remove the element added into $D$ at stage $t$ [and restore
the $\Theta $ computation]. Depending on the current stage announcement we
would then have outcome either $r$ or $w$. Since by assumption the outcome
at $s$ is $g_{\alpha _{j}}$, we must have followed either Plan S or Plan R
at $t$.
\end{proof}

Now we can prove that $\Psi $ requirements are not injured by Plan G.

\begin{lemma}
If $\sigma $ is a $\Psi $ node ($\Psi (G)\neq D$) on the true path and we go
to outcome $d$ after stage $s(\sigma )$, then the diagonalization will not
be injured thereafter, i.e., the $\Psi $-use in $G$ is preserved and the
diagonalization witness $x$ is not removed from $D$ and so the $\Psi $
requirement is satisfied.
\end{lemma}

\begin{proof}
Suppose at stage $s>s(\sigma )$ we diagonalized at $\sigma $ by putting $x$
into $D$ and so impose permanent restraint on $D$. First of all, $x$ cannot
be taken out of $D$ at subsequent stages, since only nodes above $\sigma $
could do so and then only if we follow either Plan S or Plan R at a node
above $\sigma $. Plan S action would move us to the left for a contradiction
and there are no outcomes $r$ above $\sigma $ by Lemma \ref{Plan R lem}.

Next we claim that the $G$-use is also preserved. No node below or to the
right of $\sigma $ can add elements below this $G$-use into $G$ after stage $%
s$, since their $\Delta $-uses are defined to be fresh. The only worry is
that some $\Theta $ node $\beta $ with $\beta \symbol{94}g_{\alpha }$ above $%
\sigma $ might follow Plan G (at stage $s_{1}>s$) to add $\delta (y)-1$ into 
$G$ for some changed $u$ in $W$ in order to correct some $\Delta $ axiom.
The $\Delta (G;u)$ axiom being killed must have been enumerated before stage 
$s$, since its use was chosen fresh. So $W(u)=\Phi (A;u)$ at stage $s$.

By Lemma \ref{change G lem}, the only circumstances under which we change G
at stage $s_{1}$ in response to this change in $W$ is that $D$ has already
changed by some other node $\tau $ below $\beta \symbol{94}g_{\alpha }$
following Plan S or Plan R at the previous stage $t$ at which $\beta \symbol{%
94}g_{\alpha }$ was accessible. Such a $\tau $ cannot be above $\sigma $ as
we would then move to its left, so it must be to the right of, or below, $%
\sigma \symbol{94}d$.

Then there must be another $\Psi ^{\prime }$ node $\sigma ^{\prime }$ (below 
$\tau $) which added a number into $D$ after stage $s$ (since at stage $s$
we initialized all nodes to the right of $\tau $, any change before $s$
cannot be used in shuffling or removal) and the number is taken out by $\tau 
$. However, $\sigma ^{\prime }$ cannot be below, or to the right of, $\sigma 
$, as its witness would then be larger than $s$ and so could not affect the
value of $W(u)$ which agrees with $\Phi (A;u)$ at stage $s$. So we get a
contradiction.

In another words, once we apply diagonalization for $\sigma $, we
(automatically) implement restraints for $A$ and $D$ and hence to $W$'s up
to the part that we have coded in. So we can guarantee that there can be no
change in $W$ which makes us change $G$ on the uses we have already seen,
and in particular, the $\Psi $-use of $G$ is preserved.
\end{proof}

\section{Bibliography}

%TCIMACRO{\TeXButton{parindent}{\hspace*{\parindent}}}%
%BeginExpansion
\hspace*{\parindent}%
%EndExpansion
Arslanov, M. M. [1985], Lattice properties of the degrees below $\mathbf{0}%
^{\prime }$, Doklady Ak. Nauk SSR \textbf{283}, 270-273.

Arslanov, M. M. [2009], Definability and elementary equivalence in the
Ershov difference hierarchy in\emph{\ Logic Colloquium 2006}, Lecture Notes
in Logic, Association for Symbolic Logic and Cambridge University Press, New
York, 1--17.

Arslanov, M. M. [2010], The Ershov hierarchy in \emph{Computability in
Context: Computation and Logic in the Real World}, Cooper, S. B. and Sorbi,
A. eds., Imperial College Press, London.

Arslanov, M. M., Kalimullin, I. Sh. and Lempp, S. [2010], On Downey's
Conjecture, \emph{Journal of Symbolic Logic }\textbf{75}, 401-441.

Cooper, S. B. [1971], \emph{Degrees of Unsolvability}, Ph.D. Thesis,
Leicester University.

Cooper, S. B. Harrington, L., Lachlan, A. H.. Lempp S. and Soare, R.
I.[1991], The d-r.e. degrees are not dense, \emph{Annals of Pure and Applied
Logic} \textbf{55}, 125-151.

Downey, R. G. [1989], D-r.e. degrees and the non-diamond theorem, \emph{%
Bulletin London Mathematical\ Society }\textbf{21}, 43-50.

Epstein, R. L. [1979], \emph{Degrees of Unsolvability: Structure and Theory}%
, \textbf{LNM} 759, Springer-Verlag, Berlin.

Ershov, Y. I. [1968], On a hierarchy of sets I, \emph{Algebra and Logic }%
\textbf{7}, 25-43.

Ershov, Y. I. [1968], On a hierarchy of sets II, \emph{Algebra and Logic }%
\textbf{7}, 212-232.

Ershov, Yu. I. [1970], On a hierarchy of sets III, \emph{Algebra and Logic }%
\textbf{9}, 20-31.

Gold, E. M. [1965], Limiting recursion, \emph{Journal of Symbolic Logic }%
\textbf{30}, 28-48.

Harrington, L. and Shelah, S. [1982], The undecidability of the recursively
enumerable degrees (research announcement), \emph{Bulletin of the American
Mathematical. Society}, N.S. \textbf{6}, 79-80.

Hodges, W. [1993], \emph{Model Theory}, Cambridge University Press,
Cambridge U.K.

Jockusch, C. G. Jr. and Shore, R. A. [1984], Pseudo-jump operators II:
transfinite iterations, hierarchies and minimal covers,\emph{\ Journal of
Symbolic Logic} \textbf{49}, 1205-1236.

Lachlan, A. H. [1966], Lower bounds for pairs of recursively enumerable
degrees, \emph{Proceedings of the London Mathematical. Society} (3) \textbf{%
16}, 537-569.

Lachlan, A. H. [1968], Distributive initial segments of the degrees of
unsolvability, \emph{Z. Math. Logik Grund. Math.} \textbf{14}, 457-472.

Lerman, M. [1983], \emph{Degrees of Unsolvability}, Springer-Verlag, Berlin.

Nies, A.,\ Shore, R.\ A.\ and Slaman, T.\ A.\ [1998], Interpretability and
definability in the recursively enumerable degrees, \emph{Proceedings of the
London Mathematical. Society\ }(3) \textbf{77}, 241-291.

Putnam, H. [1965], Trial and error predicates and the solution to a problem
of Mostowski, \emph{Journal of Symbolic Logic }\textbf{30}, 44-50.

Sacks, G. E. [1966], \emph{Degrees of unsolvability}, Annals of Math.
Studies \textbf{55}, Princeton University Press, $2^{nd}$ ed., Princeton NJ.

Shoenfield, J. R., [1959], On degrees of unsolvability, \emph{Annals of
Mathematics }\textbf{69, }644-653.

Shoenfield, J. R. [1965], An application of model theory to degrees of
unsolvability, in \emph{Symposium on the Theory of Models}, J. W. Addison,
L. Henkin and A. Tarski eds., North-Holland, Amsterdam, 359-363.

Shore, R. A. [1981], The theory of the degrees below $0^{\prime }$, \emph{%
Journal of the London Mathematical Society} \textbf{24}, 1-14.

Shore, R. A. [1988], Defining jump classes in the degrees below $0^{\prime }$%
, \emph{Proceedings of the American Mathematical Society} \textbf{104},
287-292.

Shore, R. A. [2006], Degree structures: Local and global investigations, 
\emph{Bulletin of Symbolic Logic }\textbf{12}, 369-389.

Shore, R. A. [2007], Direct and local definitions of the Turing jump, \emph{%
Journal of Mathematical Logic }\textbf{7}, 229-262.

Simpson, S.\ G.\ [1977], First order theory of the degrees of recursive
unsolvability, \emph{Annals of\ Mathematics\ (2)}, \textbf{105}, 121-139.

Slaman, T. A. [1983] The recursively enumerable degrees as a substructure of
the $\Delta _{2}^{0}$ degrees, handwritten notes.

Slaman, T. A. [1991], Degree structures, in Proceedings\emph{\ Int. Cong.
Math., Kyoto 1990}, Springer-Verlag, Tokyo, 303-316.

Slaman, T. A. and Woodin, H. [2001], \emph{Definability in Degree Structures}%
, preprint.

Stephan, F. Yang, Y. and Yu, L. [2009], Turing degrees and the Ershov
hierarchy in \emph{Proceedings of the Tenth Asian Logic Conference, Kobe,
Japan, 1-6 September 2008}, World Scientific, 300-321.

Taitslin, M. A. [1962], Effective inseparability of the sets of identically
true and finitely refutable formulas of elementary lattice theory, \emph{%
Algebra i Logika }\textbf{3}, 24-38.

Turing, A. M. [1939], Systems of logic based on ordinals, \emph{Proceedings
of the London Mathematical. Society }(3) \textbf{45}, 161-228.

Yang, Y. and Yu, L. [2006], \textit{$\mathcal{R}$ is not a }$\Sigma _{1}$%
\textit{-elementary substructure of $\mathcal{D}_{n}$}, \emph{Journal of
Symbolic Logic} \textbf{71}, 1223--1236.

Yates, C. E. M. [1966], A minimal pair of recursively enumerable degrees, 
\emph{Journal of Symbolic Logic} \textbf{31}, 159-168.

\end{document}
